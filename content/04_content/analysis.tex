\chapter{Comparison of MOR Methods} \label{analysis}
The previously mentioned methods for model order reduction will be compared regarding time domain error, frequency domain error and computational speed.
The time domain error will be obtained by comparing the FEM solution to a given approximation.
To get insights into the frequency domain error, the error system of a reduced order model will be analysed.
The computational speed will be determined by measuring the time it takes to generate a ROM.
Here it is assumed that the implementations provided by MORLAB are programmed in a sufficiently effective manner.
For testing the following parameters are used
\begin{gather}
\alpha = 0.1, \quad T = 1, \quad L = 1 \\
n = 100, \quad n_t = 10^{4}
\end{gather}
.

These values are chosen such that it yields results in a timely manner.
Especially \(n_t\) and \(\alpha\) are important for stability of euler scheme.
If both are too low, the euler scheme becomes unstable.
Choosing \(n_t\) too high the time and memory consumption becomes rather large.
There was no exact method for determining the parameters in this way.

\section{Time Domain Error}
The time domain error will defined as \(\epsilon = Y - \hat{Y}\) where \(Y\) and \(\hat{Y}\) denote the matrices storing the output of the systems \(G\) and \(G_r\).
Since the output matrices are usually rather large, it is impractical to use \(\epsilon\) directly.
Therefore  \(||\epsilon||_{F}\) will be considered.
The error is measured using the Frobenius norm to get a measure of the error that respects all data points.
Here two aspects are interesting.
There will be two different initial conditions considered.
The first one is \(x(0, x) = 1\).
It is chosen in this way to display the workings of the boundary condition.
If there were no boundary conditions, for a constant initial condition the system would never cool down since \(\frac{\partial^2 u}{\partial x^2} = 0\) (\ref{eq-1d-h}) at every point in time.
This only holds if there is no input to the system.
Therefore \(u(t) = 0\), where \(u\) denotes the systems input.
The second initial condition is \(x(0, x) = 4\sin(\frac{2\pi}{L}x)\) with random input.
The initial condition is sinusoidal since it is easy to approximate.
The input is random to show the error of the models if the inputs are erratic.
\pagebreak
\subsection{Proper Orthogonal decomposition}
\subsection{Modal Truncation is Equal to Balanced Truncation}
The results for BT and MT are the same since \(B = C\) and \(A\) is symmetric.
Remember that \(A^{n\times n} = M^{-1}K\) where both \(K\) and \(M\) are symmetric, hence \(A\) is symmetric \cite{170372}.
The eigenvectors of a symmetric matrix are mutually orthogonal \cite{Zhang}.
Therefore the eigendecomposition of  \(A\) is
\begin{gather}
AX = X\Delta
\end{gather}.
The eigenvalues of a symmetric matrix are real .
It is assumed that \(\Delta = diag(\lambda_1, ..., \lambda_n) \lambda_1 \geq \lambda_2 \geq ... \geq \lambda_n\).
where \(XX^T = X^TX = I\).
By applying \(x = X\zeta\) to the system, the system becomes
\begin{gather}
\dot{\zeta} = \Delta \zeta + X^{T}u(t) \label{sys-zeta1}\\
\zeta = X\zeta + Du(t) \label{sys-zeta2}
\end{gather}
The grammians of that systems are
\begin{gather}
W_c = \lim_{t \to \infty} \int_{0}^{t} e^{\Delta\tau}Ie^{\Delta\tau}d\tau \label{gramc} \\
= \lim_{t \to \infty} \int_{0}^{t} e^{2\Delta\tau}d\tau \\
= \lim_{t \to \infty} (e^{\Delta t} - I)\Delta^{-1} \\
W_o = \lim_{t \to \infty} \int_{0}^{t} e^{\Delta\tau}Ie^{\Delta\tau}d\tau \label{gramo} \\
= \lim_{t \to \infty} \int_{0}^{t} e^{2\Delta\tau}d\tau \\
= \lim_{t \to \infty} (e^{\Delta t} - I)\Delta^{-1}
\end{gather}
Therefore \(W_c = W_o\), where \(W_c\) and \(W_o\) are diagonal for stable systems.
Since the transfer of heat is a stable process, it is assumed that the according system is stable.
\begin{gather}
W_c = W_o = \lim_{t \to \infty} (e^{\Delta t} - I)\Delta^{-1} \\
= -\Delta^{-1} 
\end{gather}
\cite{658289}
\begin{gather}
W_C = W_o = \begin{bmatrix}
\frac{-1}{\lambda_1} && 0 && \hdots && 0 \\
0 && \frac{-1}{\lambda_2}&& 0 && \vdots \\
\vdots && 0 && \ddots && \vdots \\
0 && 0 && \hdots && \frac{-1}{\lambda_n}
\end{bmatrix}
\end{gather}
This shows that the eigenvectors of \(A\) satisfy the conditions for balancing transformation \(T\) in section \ref{bt}.
Since the eigenvalues are all negative and real the order \(\delta_{11} \geq \delta_{22} \geq ...  \delta_{nn}\) applies.
This enables the mode truncation described in section \ref{bt}.
\begin{gather}
\tilde{x} = \begin{bmatrix}
\zeta_1 \\
\vdots \\
\zeta_r
\end{bmatrix} \quad 
\tilde{z} = \begin{bmatrix}
\zeta_{r+1} \\
\vdots \\
\zeta_n
\end{bmatrix} \quad
z = \begin{bmatrix}
\tilde{x} \\
\tilde{z}
\end{bmatrix}\\
X= \begin{bmatrix}
X_r && X_{n-r}
\end{bmatrix} \quad
X^{-1} = \begin{bmatrix}
X_r^{*} \\
X_{n-r}
\end{bmatrix} \\
\frac{d\tilde{x}}{dt} = \Delta_r \tilde{x} + X^{-1}_ru \\
y = X_r\tilde{x} + Du 
\end{gather}
The transfer function of the system described in (\ref{sys-zeta1}) and (\ref{sys-zeta2}) can be obtained using the DCNF section \ref{dcnf}, this section also shows that the transfer function of the transformed system is equal to the transfer function of the original system.
Section \ref{mtht} shows that the optimal truncation here is to keep the terms corresponding to the \(r\) largest eigenvalues of \(A\).
The resulting system in state space representation is
\begin{gather}
\tilde{x} = \begin{bmatrix}
\zeta_1 \\
\vdots \\
\zeta_r
\end{bmatrix} \quad 
\tilde{z} = \begin{bmatrix}
\zeta_{r+1} \\
\vdots \\
\zeta_n
\end{bmatrix} \quad
z = \begin{bmatrix}
\tilde{x} \\
\tilde{z}
\end{bmatrix}\\
X= \begin{bmatrix}
X_r && X_{n-r}
\end{bmatrix} \quad
X^{-1} = \begin{bmatrix}
X_r^{*} \\
X_{n-r}
\end{bmatrix} \\
\frac{d\tilde{x}}{dt} = \Delta_r \tilde{x} + X^{-1}_ru \\
y = X_r\tilde{x} + Du 
\end{gather}
This system is equal to the system retrieved by Balanced Truncation for \(T = X\).





