\chapter{Comparison of MOR Methods} \label{analysis}
The previously mentioned methods for model order reduction will be compared regarding time domain error, frequency domain error and computational speed.
The time domain error will be obtained by compareing the FEM solution to a given approximation.
To get insights into the frequency domain error, the error system of a reduced order model will be analysed.
The computational speed will be determined by measuring the time it takes to generate a ROM.
Here it is assumed that the implementations provided by MORLAB are programmed in a sufficiently effective manner.
\section{Time Domain Error}
The time domain error will defined as \(\epsilon = Y - \hat{Y}\) where \(Y\) and \(\hat{Y}\) denote the matrices storing the output of the systems \(G\) and \(G_r\).
Since the output matrices are usually rather large, it is inpractical to use \(\epsilon\) directly.
Therefore  \(||\epsilon||_{max}\)  and \(||\epsilon||_{F}\)  will be considered.
The maximumsnorm will be used to show the magnitude of the worst occuring error.
Since this is subsceptible to spikes the error is also meassured using the Frobeniusnorm to get a meassure of the error that respects all data points.
Here two aspects are interesting.
The first one is how the errors behaves as \(r\) gets larger and the second aspect is how the error evolves over time for some fixed \(r\).
To get data about the first aspect \(||\epsilon||_{max}\)  and \(||\epsilon||_{F}\) will be meassured for increasing \(r\).
Data about the second aspect will be gathered by calculating the norm of the error between the column vectors of \(Y\) and \(\hat{Y}\) for some fixed \(r\), \(\epsilon_{t} = ||Y_t - \hat{Y}_t||_{2}\) where \(A_t\) denotes the \(t^{th}\) column vector of \(A\).
Since here vectors are compared the euclidean norm is used since it is compatible to the frobenius norm.
The chebychev norm can also be used as vector norm.

