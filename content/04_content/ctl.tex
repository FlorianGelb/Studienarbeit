\section{Fundamentals of Control Theory}
There are mainly three different problems in control theory.
A system has to be identified before it can be analyzed.
A mathematical model has to be found that relates the input of that system to its output.
After a model has been found, it is helpful to simulate the system to determine which input results in which output.
Simulations have to fit the system and offer the advantage over real experiments in reducing costs and risks attached to them.
The last of the three major problems is the control problem.
Now that the underlying model is known and the system can be simulated, designing a controller for that system is possible.
Here the controller is a system that feeds some input to the controlled system to generate some desired output \cite{DouglasBa}.
For this paper, the second problem is the most important one since the goal of this paper is to compare some methods to make simulations of a system more efficient.
\subsection{LTI Systems}
A linear time-invariant (LTI) system is a kind of system that fulfills the following conditions:
\paragraph{Homogenity:}
One of the conditions for a system to be an LTI system, is that the output of a system \(y(t)\) relates to the input \(u(t)\) and the initial condition \(x_0\) by a linear operator \(\mathfrak{f}\)
\begin{gather}
y(t) = \mathfrak{f}(u(t), x_0) \\
cy(t) = \mathfrak{f}(cu(t),  x_0) \,.
\end{gather}
This means that if the input \(u\) changes in magnitude by a constant factor of \(c\), the output \(y\) also changes by the factor of \(c\).
\paragraph{Superposition:}
The second requirement is superposition
\begin{gather}
y_1(t) + y_2(t) = \mathfrak{f}(u_1(t) + u_2(t), x_0) \,.
\end{gather}
The sum of two outputs \(y_1\) and \(y_2\) has to equal the output of the system with the sum of the inputs \(u_1\) and \(u_2\) as input.
\paragraph{Time Invariance:}
A time-invariant system has the property that if the input \(u\) is shifted in time by some constant \(\tau\), the resulting output is also shifted in time by the same constant
\begin{gather}
y(t - \tau) = \mathfrak{f}(u(t-\tau),  x_0) \,.
\end{gather}
LTI systems are important because they can be used to approximate non-LTI systems over some region.
This is useful since LTI systems are well understood \cite{DouglasB}.
\subsection{State Space Representation}
An LTI system can be expressed as a system of ODEs that relates the input \(u\) to its output \(y\).
The vector \(x\) denotes the states of a system.
\begin{align}
\dot{x} &= Ax + Bu \quad x(0) = x_0\\
y &= Cx + Du
\end{align}
The matrices \(A \in \mathbb{R}^{n \times n}\),
\(B \in \mathbb{R}^{n \times m}\),
\(C \in \mathbb{R}^{q \times n}\) and
\(D \in \mathbb{R}^{q \times m}\) are constant matrices \cite{BennerGrivet}.
\subsection{State Space of Discrete-Time Systems}
A system can be represented as a discrete system in the following form
\begin{gather}
x_{k+1} = A_dx_k + B_dx_k \label{disc-a} \\
y_{k+1} = C_dx_x + D_du_k \label{disc-b}\\
x_{k} = x_{k\Delta t} \,.
\end{gather}
The matrices \(A_d\) to \(D_d\) can be obtained from the continuous system in the following way \cite{brunton_kutz_2019g}
\begin{align}
A_d &= e^{A\Delta t}, \quad B_d = \int_0^{\Delta t} e^{A\tau}B d\tau,  \quad
C_d = C, \quad D_d = D \,.
\end{align}
\subsection{Transfer Function}
To understand the transfer function, the impulse response has to be defined first.
The impulse response of a system is given by choosing the delta function as input to the system
\begin{gather}
\delta(t) = \begin{cases}
\infty, \quad t = 0 \\
0 \quad else
\end{cases}
g(t) = f(\delta(t)) \,.
\end{gather}

It has the property that the integral of this function is 1
\begin{gather}
\int_{t = -\infty}^{\infty} \delta(t) dt = 1 \,.
\end{gather}
If the \(g(t)\) is convoluted with a function \(v(t)\) the response of the system for \(v(t)\) as input is obtained \cite{DouglasBb}
\begin{gather}
v(t)*g(t) = \int_{\tau = 0}^{\infty} v(\tau)g(t-\tau)d\tau = f(v(t)) \,.
\end{gather}
 \\
By using the convolution theorem \cite{ABELL2018399}
\begin{gather}
\mathfrak{L}\{g(t)\}\mathfrak{L}\{v(t)\} = \mathfrak{L}\{g*v\}(t) \,.
\end{gather}
\\
the definition of the transfer function is as follows
\begin{gather}
\mathfrak{L}\{f(\delta(t))\} \,.
\end{gather}
This definition is useful since the response of a system for a given input \(u(t)\) can be determined by the result of the multiplication \(G(s)U(s)\) \cite{DouglasBb}.
Alternatively, for a system in state space representation, the transfer function is \cite{BennerGrivet}
\begin{gather}
G(s) = C(sI-A)^{-1}B+D \,. \label{tf-from-ss}
\end{gather}


\subsection{Controllability and Observability}
For LTI systems in state space representation, it can be determined which states are to what degree controllable and observable.
The controllability and observability of gramians can compute this
\begin{gather}
W_c = \lim_{t \to \infty} \int_{0}^{t} e^{A\tau}BB^{*}e^{A^{*}\tau}d\tau \label{gram-ctrl}\\
W_o = \lim_{t \to \infty} \int_{0}^{t} e^{A^{*}\tau}C^{*}Ce^{A\tau}d\tau \,. \label{gram-obsv}
\end{gather}
The degree of controllability for a state \(x\) can be determined by \(x^{*}W_cx\).
If the result of this calculation is large, the system is controllable in the \(x\) direction.
By swapping \(W_c\) with \(W_o\) the degree of observability in state \(x\) can be computed: \(x^{*}W_ox\).
Again if the resulting value is large, the system can be observed well in the state \(x\) \cite{brunton_kutz_2019d}.

