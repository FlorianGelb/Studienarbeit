\chapter{Fundamentals}
This chapter covers the fundamentals necessary for understanding the model order reduction methods. First of all, a system to apply those methods is needed. Here the 1D heat equation was chosen since it is fairly easy to solve and to understand the results. After that, the method of finite elements is introduced. This method discretizes the 1D heat equation into a solvable system of ODEs since it can handle boundary conditions and initial values. One of the most important fundamentals of this paper is the singular value decomposition since it takes heavy use in the model order reduction methods covered. The last Section of this chapter will cover some control theory fundamentals. Those are needed since three model order reduction methods appear in a control theory context. Hence some explanation of the terminology is needed.
\section{Heat Equation} \label{HE}
The following partial differential equation (PDE) describes the conduction of heat  within a medium
\begin{gather}
\frac{\partial u}{\partial t} = \alpha \nabla^{2} u\,. \label{heat}
\end{gather}
Here \(u\) is a function of space and time, and \(\alpha\) is a positive constant.
For this paper \(u\) will be defined in terms of one spatial dimension \cite{Gustafsson2011}
\begin{gather}
u := u(x, t) \\
\frac{\partial u}{\partial t} = \alpha \frac{\partial^{2} u}{\partial x^{2}} \\
x \in \chi \subset \mathbb{R} \quad t \in \tau \subset \mathbb{R} \\
x_{0} \leq x \leq x_{n} \quad t_{0} \leq t \leq t_{n}\,.
\end{gather}

In order to model the conduction of heat within a medium but also a heating process, a new function \(h: \chi \times \tau \rightarrow  \mathbb{R}\) is introduced:
\begin{gather}
\frac{\partial u}{\partial t} = \alpha \frac{\partial^{2} u}{\partial x^{2}} + h(x,t)\,. \label{eq-1d-h}
\end{gather}
For this paper, the assumption is made that the initial condition is known
\begin{gather}
f: \chi \rightarrow \mathbb{R} \\
u(x, t_{0}) = f(x)\,.
\end{gather} 

Applying the Fourier transform (FT) w.r.t \(x\) to (\ref{eq-1d-h}) yields the inhomogeneous ordinary differential equation (ODE)
\begin{gather}
\hat{u} = \mathfrak{F}(u) \quad \hat{h} = \mathfrak{F}(h) \\
\frac{d}{dt} \hat{u} = -\alpha\omega^{2}\hat{u} + \hat{h}\,. \label{eq-1d-h-ft}
\end{gather}

A solution to (\ref{eq-1d-h-ft}) is given by
\begin{gather}
\hat{u} = \hat{u}_{0} + \hat{u}_{p}\,.
\end{gather}
Where \(\hat{u}_{0}\) is the homogeneous solution and \(\hat{u}_{p}\) is the particular integral.
In order to solve this ODE, the particular integral \(\hat{h}\) has to be known \cite{Papula2015}.
The choice of \(h\) is, except for some restrictions, arbitrary.
Therefore an approximate solution to \ref{eq-1d-h-ft} \(\hat{u}_{a}\) is obtained by the forward Euler scheme
\begin{gather}
\frac{d}{dt} \hat{u} \approx \frac{\Delta \hat{u}}{\Delta t} \\
\hat{u}_{t+1} = \hat{u}_{t} + \Delta t (-\alpha\omega^{2}\hat{u} + \hat{h}) \label{eq-1d-h-es} \\
\hat{u}_{a} = \begin{bmatrix}
\hat{u}_{t_{0}} & \hdots & \hat{u}_{t_{n}}
\end{bmatrix}\,.
\end{gather}
The Euler scheme is applicable if an initial condition \(\hat{u}_{0}\) is known. 
This initial condition is obtained by applying the discrete Fourier transform (DFT) to an initial temperature distribution along \(x\) \cite{Gustafsson2011b}
\begin{gather}
\hat{u}_{0} = \mathfrak{F} \{f(x)\}\,.
\end{gather} 


The forward Euler scheme is used because it is easy to implement.
Applying the inverse discrete Fourier transform (IDFT) to \(\hat{u}_{a}\), an approximate solution to (\ref{eq-1d-h}) can be obtained.
Using fast Fourier transform (FFT) and inverse fast Fourier transform (IFFT)  instead of the DFT, and IDFT reduces the processing time.
This method for solving the heat equation raises the problem that it cannot handle boundary conditions. 
Using the finite element method, the heat equation is solvable as an initial boundary value problem (IBVP).
