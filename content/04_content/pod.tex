\section{Proper Orthogonal Decomposition}
The proper orthogonal decomposition (POD) is a method for model order reduction.
The reduction in computational effort is done by approximating a solution to a PDE using an orthogonal expansion.
The basis functions are obtained by decomposing a set of solutions to othe PDE using the SVD.
\subsection{Orthogonal Expansion}
A function \(f: \mathbb{R} \times \mathbb{R} \rightarrow \mathbb{R}\) can be represented by the series:
\begin{gather}
f(x, t) = \sum_{i = 1}^{\infty}a_i(t)\phi_i(x) \\
x, t \in \mathbb{R} \label{ref-orth-exp}
\end{gather}
Here all \(\phi(x)\) are orthorgonal basis functions.
\begin{gather}
\langle\phi_i, \phi_j\rangle =\begin{cases}
1, \quad \text{if } i = j \\
0, \quad \text{else}
\end{cases} \label{phi-orth}
\end{gather}
By using a finite sum instead of the entire  series \(f\) can be approximated:
\begin{gather}
\tilde{f}(x, t) = \sum_{i = 1}^{n}a_{i}(t)\phi_{i}(x) \label{ref-orth-aprox}
\end{gather}
Since all \(\phi\) are known, \ref{ref-orth-aprox} has to be solved for the set \(\{a_0, ..., a_n\}\).
In case \(f\) is known, the solutions contained in this set can be calculated as:
\begin{gather}
a_i(t) = \frac{\langle \phi_i, f \rangle}{\langle \phi_i, \phi_i \rangle} \label{sol-ai}
\end{gather}
\cite{Gustafsson2011e}
\subsection{Proper Orthogonal Decomposition for PDEs}
Since a PDEs are equations in terms of partial derivatives, the notation \(\mathscr{P}(\partial x) u\) is introduced, which denotes a differential operator in terms of the spacial variables \(x\) for a function \(u(x,t,p\).
The vector \(p\) contains some parameters.
This is done to provide a more abstract way to denote PDEs:
\begin{gather}
\frac{\partial u}{\partial t} = \mathscr{P}(\partial x) u
\end{gather}
\cite{Gustafsson2011f}
Solving for \(u\) is often dificult to imposible.
A method that is often used to solve PDEs is called seperation of variables.
This spereation of variables assumes, that the underlying solution \(u(x, t)\) can be expressed by \ref{ref-orth-exp}, to solve for \(a_k, 0 \leq k \leq n\).
Since it is not practical to compute an infite series, \(u\) only gets appriximated by using \ref{ref-orth-aprox} instead:
\begin{gather}
\sum_{i = 1} ^{n} \frac{\partial a_i}{\partial t} \phi_i = \sum_{i = 1} ^{n} \mathscr{P}(\partial x) \phi_i a_i \label{label-u-aprox} 
\end{gather}
By discretizing the spacial dimension got along \(x\) \ref{label-u-aprox} can be expressed using marix notation:
\begin{gather}
\Phi = \begin{bmatrix}
\phi_0, ..., \phi_n
\end{bmatrix} \label{mat-phi}\\
\Phi \frac{d}{dt}a = \mathscr{P}(\partial x) \Phi a
\end{gather}
Since the solution \(u\) is unknown \ref{sol-ai} cannot be computed.
However the fact, that all \(\phi\) are orthogonal to each oter, can be used to solve for all \(a_k\).
It can be done by computing the inner product of \ref{label-u-aprox} with all basis funktions:
\begin{gather}
\langle \sum_{i = 1} ^{n} \frac{\partial a_i}{\partial t} \phi_i, \phi_k \rangle = \langle\sum_{i = 1} ^{n} \mathscr{P}(\partial x) \phi_i a_i, \phi_k \rangle \quad \forall 0 \leq k \leq n \label{u-galer}
\end{gather}
This resembles the Galerking projection.
In matrix notation it can be expressed:
\begin{gather}
\Phi^{*}\Phi \frac{d}{dt}a = \Phi^{*}\mathscr{P}(\partial x) \Phi a
\end{gather}
By considering \ref{phi-orth}  the equiation \ref{u-galer} can be expressed as a system of ODEs which can be solved:
\begin{gather}
\frac{d}{dt} a = \Phi^{*}\mathscr{P}(\partial x) \Phi a
\end{gather}
After the vector of coefficiants \(a\) for each time step has been computed, the solution can be assembled:
\begin{gather}
u(x, t) \approx \Phi(x)a(t) \label{u-aprox-pod}
\end{gather} 
\cite{brunton_kutz_2019c}
\subsection{Choosing basis vectors}
As discussed in the previous section, a set of basis vectors can be used to generate approximate solutions to PDEs.
However it was not discussed how those basis functions are chosen.
For POD to work, a so called snapshot matrix \(X\) has to be available.
This snapshot matrix stores a set of solutions where \(x_k\) is the solution for a PDE at time step \(k\Delta t\):
\begin{gather}
X = \begin{bmatrix}
x_1, ..., x_m
\end{bmatrix}
\end{gather}
The solutions can be obtained by conducting an expirment on a physical system that is described by the PDE or by simulating the evolution of that PDE.
In this paper the solution is obtained using FEM \ref{FEM}.
SVD is used to decompose the snapshot matrix \(X\).
Since the columns of the snapshot matrix contain spacial information at a given point in time and the basis vectors are supposed to encode the spacial information of a solution \(u\) the \(r\) most dominant left singular values have to be extracted:
\begin{gather}
\tilde{X} = \tilde{U}\tilde{\Sigma}\tilde{V}^{*} \\
\Phi = [u_1, ..., u_r] \label{PHI}
\end{gather}
Here the FEM solution is obtained by discretizing the PDE into a large number(\(n\)) of spacial nodes.
This results in a high dimensional system of ODEs.
Since \(\Phi\) contains only \(r\) column (\(r << n\)) vectors a reduction in order can be achieved.
However the benefits of this reduced order model are only relevant after the PDE has been solved once using a high dimensional system of ODEs.
\cite{brunton_kutz_2019c}
\subsection{ROM for heat equation}
In order to apply pod to heat equation a snapshot matrix \(X\) has to be generated.
This is done by the FEM solver described in \ref{FEM}.
After that \(X\) gets decomposed using the SVD.
The modes contained in \(\Phi\) is obtained by truncating the left singular vectors of \(X\) according to \ref{PHI}.
Substituting \(\mathscr{P}\) in \label{label-u-aprox} with heat equation \ref{eq-1d-h} results in the following:
\begin{gather}
\Phi \frac{\partial a}{\partial t}  = \alpha \frac{\partial^{2} \Phi}{\partial x^{2}} a + h\\
\frac{\partial a}{\partial t} = \alpha \Phi^{*}  \frac{\partial^{2} \Phi}{\partial x^{2}} a + \Phi^{*}h
\end{gather}
\newpage
This system of ODEs can now be solved using a Runge-Kutta scheme.
Note that \(\Phi\) the contains numeric values.
Therefore derivatives are unstable, especially at the first and last entries of the column vectors of \(\Phi\).
To reduce this problem the method used to compute the second derivative of \(\Phi\) is the so called spectral derivative.
The discrete spectral derivative works by computing the FFT of a vector.
That vector gets multiplied by \((ik)^{d}\), \(d\) is the order of the derivative and \(k\) are the discrete wave numbers.
After that step the IFFT is applied to obtain the derivative of the original vector.
\begin{gather}
f \in \mathbb{C}^{n}, \quad k = [-\frac{n}{2} \hdots \frac{n}{2}]^{T} \\
\frac{df}{dx} = \mathfrak{F}^{-1}\{i \frac{2 \pi k}{n} \mathfrak{F}\{f\}\} \\
\frac{d^{2}f}{dx^{2}} = \mathfrak{F}^{-1}\{-\frac{2 \pi k}{n} \mathfrak{F}\{f\}\} \\
\end{gather}
\cite{brunton_kutz_2019f}


