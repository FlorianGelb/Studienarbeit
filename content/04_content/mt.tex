\section{Modal truncation}
\subsection{Diagonal canonical form}
A LTI system \((A, B, C, D)\) with transfer function \(G(s)\) can be written in the following way:
\begin{gather}
I = \{1, 2, \hdots, n\} \\
G(s) = D + \sum_{i \in I} \frac{\Phi_i}{s-\lambda_i} \label{dcnf}
\end{gather}

The set \(\Lambda = \{\lambda_1, \hdots, \lambda_n\}\) denotes the eigenvalues of the matrix \(A\) in state space.
This resembles the partial fraction decomposition of \(G(s)\) \cite{vuillemin2020optimal}.
The residue \(\Phi\) can be computed using the eigendecomposition of \(A\) and using the eigenvectors for a coordinate transform:
\begin{gather}
AX = X\Delta \quad X \zeta =  x 
\end{gather}
The system matrices have to be transformed accordingly:
\begin{gather}
\hat{A} = X^{-1}AX = \Delta \\
\hat{B} = X^{-1}B \quad \hat{C} = CX \\
\hat{D} = D
\end{gather}
Those transformed matrices are used to calculate the transfer function \ref{tf-from-ss}:
\begin{gather}
G(s) = CX(sI - \Delta)^{-1}X^{-1}B + D\\
= D + \begin{bmatrix}
C x_1 && \hdots && Cx_n
\end{bmatrix} diag\{\frac{1}{s-\lambda_1}, \hdots, \frac{1}{s-\lambda_n}\} \begin{bmatrix}
x_1^{-1}B \\
\vdots \\
x_n^{-1}B
\end{bmatrix} \\
= D + \sum_{i \in I} \frac{c_i b_i^{T}}{s - \lambda_i} 
\Rightarrow \Phi_i = c_i b_i^{T}
\end{gather}
\cite{Benner}

\subsection{Optimal modal truncation}
The goal of modal truncation is to find a subset of the indices \(I\)  such that only \(r\) elements are contained in this subset:
\begin{gather}
I_r \subseteq I, \quad |I_r| = r
\end{gather}
Using this subset as indices in \ref{dcnf} a truncation is obtained, yielding a system defined by a new transfer function \(\hat{G}(s)\).
The set \(I_r\) has to be chosen such that the error \(||G(s) - \hat{G(s)}||_{H_n}\) becomes minimal.
Here  \(H_n\) denotes either the \(H_2\) or \(H_{\infty}\) norm.
As shown in \cite{vuillemin2020optimal} other norms are also usable but for simplicity only the two stated norms are used.
This yields the following optimization problem:
\begin{gather}
\min_{\alpha} G_{\alpha}(s) = D_{\alpha} + \sum_{i \in I} \alpha_i \frac{\Phi_i}{s - \lambda_i} s.t.\\
\alpha^{T}\alpha = r \\
J = \{(\lambda_i, \lambda_j) \in I | \lambda_i \in \mathbb{C}, Im(\lambda_i) > 0, \lambda_i = \bar{\lambda_j}\}
\end{gather}
Where \(\alpha \in \{0, 1\}^{n}\) is some binary vector with \(\alpha^{T} \alpha = r\).
In case some \(\lambda_i \in \Lambda\) is a complex number and \(i \in I_r\), the index of the according complex conjugate eigenvalue has to selected too, if \(Im(\lambda_i) > 0)\) \(\lambda_j = \bar{\lambda}_i \in \Lambda \wedge i \in I_r \Rightarrow j \in I_r\). 



