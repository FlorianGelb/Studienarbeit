\chapter{Introduction}
With the development of semiconductor-based computers, humanity has access to the most powerful computing machines ever created. Still, those cutting-edge technologies often need to provide more computational power to solve problems in maths, physics, or engineering. Often those problems appear as partial differential equations. Discretizing PDEs into systems of ordinary differential equations allows the generation of approximate solutions. This resulting system of ODEs is often still too large to use for computations directly. Methods to reduce the size of those systems are called model order reduction methods.

\section{Problem statement}
Various model reduction methods can be chosen based on the problem and its setting. Each of them approximates the original system in some optimal manner. This paper aims to compare the properties of a defined set of model order reduction methods. The compared properties are the error of the reduced order models and the time it takes to compute them. The system that gets reduced describes the heat transfer in a piece of wire that can be heated or cooled arbitrarily.

\section{Outline}
The fundamentals are covered first. It begins with the basics of the heat equation and how the already mentioned heating or cooling works, respectively. Also, some naive method to obtain an approximate solution is covered. However, the next part of that chapter will discuss the finite element method since this technique is more powerful than the previously stated method.
After that, the singular value decomposition is covered since it sees much usage in the model order reduction methods. Since most methods come from control theory, the subsequent section covers some basic control theory. Chapter three is about the workings of the model order reduction methods. It begins with the proper orthogonal decomposition. Balanced truncation, modal truncation, and Hankel norm approximation follow.
Last but not least, it gets demonstrated that modal truncation equals balanced truncation for this system. In chapter four, the implementation of those methods and the implementation of the finite element solver is discussed. After that, the conduction of numerical experiments
takes place. Here the time domain error, frequency domain error, and the time it takes to compute the reduced order models are measured and compared.