\section{Finite Element Method} \label{FEM}
The finite element method (FEM) is a method to approximate solutions for differential equations (DE) within a certain domain \(\Omega\).
This is done by discretizing the spatial domain.
Assume that a DE is given by:
\begin{gather}
m, n \in \mathbb{N} \quad \zeta \in \Omega \subset \mathbb{R} \quad m \geq 1 \\
\frac{\partial^{m} y}{\partial \zeta^{m}} -  g(y) = r(\zeta, t) \label{eq-pde-gen} 
\end{gather}
It is assumed that \(g\) is a linear function that can also contain partial derivatives of \(y\) w.r.t. time, \(y\) takes the value 0 at the boundary \(\Gamma\) and \(y(\zeta, 0) = f(\zeta)\).
An approximate solution to \(y\) is given by \(\mu\), which is expressed as a sum of basis functions contained in the set \(\phi\):
\begin{gather}
\mu(\zeta, t) = \sum_{j = 1}^{N} c_{j}(t)\phi_{j}(\zeta) \label{eq-aprox}
\end{gather}
The residual is defined as:
\begin{gather}
\mathfrak{r} = \frac{\partial^{m} \mu}{\partial \zeta^{m}} -  g(\mu) - r(\zeta, t) 
\end{gather}
Furthermore the residual is required to be orthogonal to all basis functions:
\begin{gather}
\langle \mathfrak{r}, \phi_{k} \rangle = 0 \quad \forall \phi_{k} \in \phi \label{eq-req-orth}
\end{gather}

Since the functions in \(\phi\) are known, it is only required to find the coefficients \(c_{j}(t)\) in \ref{eq-aprox}.
To find those coefficients \ref{eq-req-orth} needs to be expressed as follows:
\begin{gather}
\int_{\Omega} \frac{\partial^{m} \mu}{\partial \zeta^{m}} \phi_{k} \, d\zeta  -  \int_{\Omega} g(\mu) \phi_{k}        \, d\zeta = \int_{\Omega}  r(\zeta, t) \phi_{k}        \, d\zeta \quad \forall \phi_{k} \in \phi 
\end{gather}
If \(\mu\) is substituted with \ref{eq-aprox} the following is obtained:
\begin{gather}
\sum_{j = 1}^{N} ((\int_{\Omega} \frac{\partial^{m} \phi_{j}}{\partial \zeta^{m}} \phi_{k} \, d\zeta) c_{j}(t) - g((\int_{\Omega} \phi_k \phi_j d\zeta) c_{j}(t)))  = \int_{\Omega}  r(\zeta, t) \phi_{k}        \, d\zeta \quad \forall \phi_{k} \in \phi \label{eq-al}
\end{gather}

It is also necessary to apply the divergence theorem to the first integral term taking into account that \(y\) at \(\Gamma\) is 0.
Since \(\zeta\) is one dimensional, the divergence theorem becomes integration by parts:
\begin{gather}
\int_{\Omega} \frac{\partial^{m} \phi_{j}}{\partial \zeta^{m}} \phi_{k} \, d\zeta = - \int_{\Omega} \frac{\partial^{m-1} \phi_{j}}{\partial \zeta^{m-1}} \frac{\partial \phi_{k}}{\partial \zeta} \, d\zeta \quad \forall \phi_{k} \in \phi \label{eq-ibp}
\end{gather}
Combining \ref{eq-al} and \ref{eq-ibp} yields:
\begin{gather}
-\sum_{j = 1}^{N} ((\int_{\Omega} \frac{\partial^{m-1} \phi_{j}}{\partial \zeta^{m-1}} \frac{\partial \phi_{k}}{\partial \zeta} \, d\zeta) c_{j}(t) + g((\int_{\Omega} \phi_k \phi_j d\zeta) c_{j}(t)))  = \int_{\Omega}  r(\zeta, t) \phi_{k}        \, d\zeta \quad \forall \phi_{k} \in \phi \label{eq-fem}
\end{gather}

This formulation leads to a system of ODEs or a system of linear equations that can be solved either analytically or numerically.
\subsection{Solving the Heat Equation using FEM}
This formulation of FEM can be applied to \ref{eq-1d-h}:
\begin{gather}
\Omega = \chi \quad \Gamma = \{x_{0}, x_{n}\} \\
y(\zeta, t) = -u(x, t) \quad g(u) = -\frac{1}{\alpha} \frac{\partial u}{\partial t} \\
m = 2 \quad r(\zeta, t) = \frac{1}{\alpha} h(x,t) \\
u(x, 0) = f(x) \quad u(x_{0}, t) = 0 \quad u(x_{n}, t) = 0
\end{gather}
The set of basis functions is defined as a set of piecewise linear functions with constant step size \(\Delta x\):
\begin{gather}
    \phi_j(x)= 
\begin{cases}
    (x - x_{j-1}) / \Delta x), \quad x_{j-1} \leq x <  x_{j}\\
    (x_{j+1} - x) / \Delta x), \quad x_{j} \leq x <  x_{j + 1}\\
    0,              \quad \text{otherwise}
\end{cases}
\end{gather}
\cite{Gustafsson2011d}


The step size \(\Delta x\) is defined by \(\Delta x = \frac{x_n - x_0}{n-1}\). \label{def-delta-x}
This results in the following system of ODEs:
\begin{gather}
\sum_{j=1}^N(\int_{\chi} \phi_{j}\phi_{k}dx)\frac{dc_{j}}{dt} = \alpha \sum_{j = 1}^N(-\int_{\chi} \frac{d\phi_{j}}{dx}\frac{d\phi_{j}}{dx}dx)c_{j}(t) + \int_{\chi}h(x, t) \phi_{k} dx
\quad \forall \phi_{k} \in \phi \label{eq-heat-fem}
\end{gather}
Since \(\phi_k \phi_j \neq 0\) for \(k = j, k = j \pm 1\) (\ref{eq-heat-fem}) 
can be expressed as 
\begin{gather}
q_1\dot{c}_{j-1} + q_2 \dot{c}_{j} + q_1 \dot{c}_{j+1} = p_1 c_{j-1} + p_2 c_{j} + p_1 c_{j+1} + H(x,t)
\end{gather}
where \(p\) and \(q\) are some constants derived later.
For \(j=1\)  and \(j=n\) some new coefficients \(c_0, c_{n+1}\) are introduced.
To force the boundary condition they are set to zero \(c_0 = c_{n+1} = 0\).
It follows that \(h(x, t)\) and \(c_j(0)\) have to satisfy the boundary condition too.

Using matrix notation this becomes:
\begin{gather}
M^{N \times N}, K^{N \times N} \\
M\dot{c} = Kc + d \label{eq-heat-almost-ss}
\end{gather}
The matrices \(M\) and \(K\) can be easily computed (Appendix \ref{ap-mat-der}):
\begin{gather}
m_{ij} = \begin{cases}
\frac{2\Delta x}{3}, \quad k = j \\
\frac{\Delta x}{6}, \quad |k - j| = 1 \\
0, \quad otherwise 
\end{cases} \label{def-mat-a}
\quad
k_{ij} = \begin{cases}
\frac{-2\alpha}{\Delta x}, \quad k = j \\
\frac{\alpha}{\Delta x}, \quad |k - j| = 1 \\
0, otherwise
\end{cases}
\end{gather}
Furthermore to solve this system of ODEs numerically an initial condition \(c_{0}\) has to be known 
\cite{Gustafsson2011b}.
It can be obtained using a least squares (LS) approach:
\begin{gather}
\sum_{j = 1}^N \langle \phi_j, \phi_k \rangle c_{j}(0) = \langle f, \phi_{k} \rangle \quad \forall \phi_k \in \phi \\
Mc_{0} = F \label{F}\\ 
c_{0} = M^{-1}F  \label{c0}
\end{gather}
\cite{Gustafsson2011c}.
However since the basis functions are piecewise linear functions, the coefficients are \(c_0 = \begin{bmatrix}
f(0), f(\Delta x), ..., f(n)
\end{bmatrix}^T\)
\cite{Gustafsson2011g}.
It is necessary to approximate \(d\) for each point in time using numerical integration schemes.
By assuming that the following decomposition holds:
\begin{gather}
h(x, t) = \sum_{i = 1}^{N} \phi_i(x) \upsilon(t)
\end{gather}
the vector \(d\) in \ref{eq-heat-almost-ss} can be expressed as \(M\upsilon(t)\).
Observe that by multiplying \ref{eq-heat-almost-ss} with \(M^{-1}\) (\ref{ap-K-inv}) yields a system of ODEs:
\begin{gather}
\dot{c} = M^{-1}Kc + M^{-1}M\upsilon \label{almost-almost-ss}
\end{gather}
This system of ODEs can be solved using an euler scheme:
\begin{gather}
c_{t+1} = \Delta t (M^{-1}Kc_{t} + I\upsilon) + c_{t} \label{fem-euler}
\end{gather}

Using \ref{eq-aprox} and the computed coefficients \(c\) the function \(u(x,t)\) can be approximated.
However this is equivalent to linear interpolation between \(c_{n}\) and \(c_{n+1}\) (Appendix \ref{ap-lin-interp}).
