\section{Hankel Norm Approximation}
The Hankel Norm Approximation (HNA) aims to find a reduced order system that approximates a system \(G\) such that the error in the so-called Hankel norm becomes minimal.
The Hankel norm is defined to be the largest Hankel singular value \cite{singh}
\begin{gather}
||G(s)||_H = \sigma_{max} = \sqrt{\lambda_{max}(W_cW_o)} \,.
\end{gather}
Therefore the problem can be stated as follows
\begin{gather}
G_r = \argmin ||G - G_r||_H \,. \label{hna-prob}
\end{gather}
\subsection{System Spaces}
For the following section it is necessary to define the four following system spaces: \(L_{\infty}\), \(H_{\infty}\), \(H_{\infty}^-\) and \(H_{\infty}^-(r)\).
\paragraph{\(L_{\infty}\)}
\(G \in L_{\infty}\) iff  \(\sup_{\omega}||G(i\omega)|| < \infty\).
\paragraph{\(H_{\infty}\)}
\(G \in H_{\infty}\) iff  \(\forall \lambda \in \mathbb{C}_{-}\).
Here \(\lambda\) are the eigenvalues of \(A\).
\paragraph{\(H_{\infty}^-\)}
\(G \in H_{\infty}^-\) iff \(G(-s) \in H_{\infty}\).
\paragraph{\(H_{\infty}^-(r)\)}
\(G \in H_{\infty}^-(r)\) iff \(G \in L_{\infty}\) and 
\(\lambda = \lambda_+ \cap \lambda_- = \{\lambda_1, ..., \lambda_n \}\) with \(\lambda_{\circ} = \{\lambda_i \in \lambda | \lambda_i \in \mathbb{C}\_{\circ}\}\) and \(|\lambda_-| \leq r\).


\subsection{Optimal Solution}
A lower bound for \(\min ||G - G_r||_H\) is established by lemma 7.1 in \cite{glover84}
\begin{gather}
\min ||G - G_r||_H \geq \sigma_{r+1} \,.
\end{gather}

Here \(\sigma_{r+1}\) is the \(r+1\)th largest Hankel singular value of \(G\).
Hence a system \(G_r\) is optimal if \(||G - G_r||_H = \sigma_{r+1}\).

It is important to note that the Hankel norm can is related to the \(L_{\infty}\) norm through the Nehari Theorem
\begin{gather}
G \in H_{\infty}, \quad F \in H_{\infty}^-, \quad G - F \in L_{\infty} \\
||G||_H = \min_{F \in H^{-}_{\infty}} ||G - F||_{\infty} \,.
\end{gather}

Also, the Adamjan-Arov-Krein theorem has to be stated to find an optimal solution to the stated minimization problem
\begin{gather}
G \in H_{\infty}, \quad Q \in H_{\infty}^-(r), \quad G - Q \in L_{\infty} \\
\min_{Q \in H_{\infty}^{-}} ||G-Q||_{\infty} = \sigma_{r+1} \,.
\end{gather}

Suppose there is an optimal system \(Q^{*} = G_r + F\) with \(Q^{*}  \in H_{\infty}^-(r), G_r \in H_{\infty}, F \in H_{\infty}^-\).
It has the following error bound
\begin{gather}
||G - G_r||_{\infty} = ||G - Q^{*} + F||_{\infty} \leq \sigma_{r+1} + ||F||_{\infty}\,. \label{fmin}
\end{gather}
If \(||F||_{\infty}\) is small enough, the stable part of \(Q^{*}\) can be used as a reduced order model.
It is also an optimal solution to (\ref{hna-prob}) 
\cite{sandberg}
\begin{gather}
||G - G_r||_H = \min_{F \in H^{-}_{\infty}} ||G - G_r - F||_{\infty} = \min_{Q \in H_{\infty}^{-}} ||G-Q||_{\infty} = \sigma_{r+1} \,.
\end{gather}

\subsection{Constructing \(Q^{*}\)}
The first step to construct an optimal system \(Q^{*}\) is to construct a balanced realization of the system \(G = (A, B, C, D) \in H_{\infty}\)that is to be reduced, as described in section \ref{balre}.
Hence the gramians \(W_c\) and \(W_o\) are equal and diagonal
\begin{gather}
W_c = \begin{bmatrix}
P_1 & 0 \\
0 & \sigma_{r+1}I_l
\end{bmatrix}, \quad
W_o = \begin{bmatrix}
Q_1 & 0 \\
0 & \sigma_{r+1}I_l
\end{bmatrix} \,. 
\end{gather}
Now \(G\) is partitioned in the following way
\begin{gather}
A = \begin{bmatrix}
A_{11} & A_{12} \\
A_{21} & A_{22}
\end{bmatrix}, \quad 
B = \begin{bmatrix}
B_{1}  \\
B_{2} 
\end{bmatrix}, \quad 
C = \begin{bmatrix}
C_{1}  \\
C_{2} 
\end{bmatrix} \,.
\end{gather}

Also a unitary matrix \(U\) has to be defined such tat \(B_2 = -C_2^TU\) and \(U^TU = I\). 
Further more a matrix \(E_1 = P_1Q_1-\sigma_{r+1}^2I\) is introduced.

Then a system \(Q^* = (\hat{A}, \hat{B}, \hat{C}, \hat{D})\) can be defined \cite{sandberg}
\begin{align}
\hat{A} &= E_1^{-1}(\sigma_{r+1}^2A_{11}^T + Q_1 A_{11}P_1 - \sigma_{r+1}C_1^TUB_1^T) \\
\hat{B} &= E_1^{-1}(Q_1B_1 + \sigma_{r+1}C_1^TU)\\
\hat{C} &= C_1P_1 + \sigma_{r+1}UB_1^T \\
\hat{D} &= D - \sigma_{r+1}U \,.
\end{align}

\subsection{Decomposition of \(Q^{*}\)}
The final step is to decompose \(Q^{*}\) into two systems \(G_r \in H_{\infty}, F \in H_{\infty}^{-}\).
To achieve this, \(Q^{*}\) can be expressed in the diagonal canonical form described in section \ref{secdcnf}.
\begin{gather}
Q^{*} = D + \sum_{i=1}^{r} \frac{\phi_i}{s-\lambda_i} \,.
\end{gather}

This can no be decomposed into three systems \(K \in H_\infty, V \in H_{\infty}^-\) and \(\tilde{D}\) with
\begin{gather}
\operatorname{Re}(\lambda_i) \in \mathbb{C}_{\circ} \forall i \in I_{\circ} \\
K = \sum_{i \in I_-} \frac{\phi_i}{s-\lambda_i} \\
V = \sum_{i \in I_+} \frac{\phi_i}{s-\lambda_i} \\
\tilde{D} = \hat{D} \,.
\end{gather}

Since \(\tilde{D}\) is some constant, it does not have any poles.
Therefore \(K + D \in H_\infty\) and \(V + D \in H_{\infty}^-\) which leads to two different decompositions \(Q^{*} = (K + \tilde{D}) + V \) and \(Q^{*} = K + (V + \tilde{D})\).
From (\ref{fmin}), it is clear that \(||F||_{\infty}\) has to be as small as possible to minimize the error bound.
It can be shown that \(Q^{*} = (K + \tilde{D}) + V \) is the optimal decomposition.
Suppose \(Q^{*} = K + (V + \tilde{D})\) was the optimal decomposition, then the corresponding lower bound has to be smaller
\begin{align}
||G-K||_{\infty} &\leq ||G-(K+\tilde{D})||_{\infty} \\
\sigma_{r+1}||V + \tilde{D}||_{\infty} &\leq \sigma_{r+1}||V||_{\infty} \\
||V + \tilde{D}||_{\infty} &\leq ||V||_{\infty} + ||\tilde{D}||_{\infty} \geq ||V||_{\infty} \\
\Rightarrow ||G-K||_{\infty} &\geq ||G-(K+\tilde{D})||_{\infty} \,.
\end{align}
Therefore \(G_r = K + D\) and \(F = V\).

\section{Modal Truncation is Equal to Balanced Truncation}
The results for BT and MT are the same since \(B = C\) and \(A\) is symmetric.
Remember that \(A^{n\times n} = M^{-1}K\) where both \(K\) and \(M\) are symmetric, hence \(A\) is symmetric \cite{170372}.
The eigenvectors of a symmetric matrix are mutually orthogonal \cite{Zhang}.
Therefore the eigendecomposition of  \(A\) is
\begin{gather}
AX = X\Delta \,.
\end{gather}
The eigenvalues of a symmetric matrix are real.
It is assumed that \(\Delta = diag(\lambda_1, ..., \lambda_n)\) with \(\lambda_1 \geq \lambda_2 \geq ... \geq \lambda_n\).
where \(XX^T = X^TX = I\).
By applying \(x = X\zeta\) to the system, the system becomes
\begin{align}
\dot{\zeta} &= \Delta \zeta + X^{T}u(t) \label{sys-zeta1}\\
\zeta &= X\zeta + Du(t) \,. \label{sys-zeta2}
\end{align}
The gramians of that system are
\begin{align}
W_c &= \lim_{t \to \infty} \int_{0}^{t} e^{\Delta\tau}Ie^{\Delta\tau}d\tau \label{gramc} \\
&= \lim_{t \to \infty} \int_{0}^{t} e^{2\Delta\tau}d\tau \\
&= \lim_{t \to \infty} (e^{\Delta t} - I)\Delta^{-1} \\
W_o &= \lim_{t \to \infty} \int_{0}^{t} e^{\Delta\tau}Ie^{\Delta\tau}d\tau \label{gramo} \\
&= \lim_{t \to \infty} \int_{0}^{t} e^{2\Delta\tau}d\tau \\
&= \lim_{t \to \infty} (e^{\Delta t} - I)\Delta^{-1} \,.
\end{align}
Therefore \(W_c = W_o\), where \(W_c\) and \(W_o\) are diagonal for stable systems.
Since the transfer of heat is a stable process, it is assumed that the according system is stable \cite{658289}
\begin{gather}
W_c = W_o = \lim_{t \to \infty} (e^{\Delta t} - I)\Delta^{-1} \\
= -\Delta^{-1} 
\end{gather}

\begin{gather}
W_C = W_o = \begin{bmatrix}
\frac{-1}{\lambda_1} && 0 && \hdots && 0 \\
0 && \frac{-1}{\lambda_2}&& 0 && \vdots \\
\vdots && 0 && \ddots && \vdots \\
0 && 0 && \hdots && \frac{-1}{\lambda_n}
\end{bmatrix} \,.
\end{gather}
This shows that the eigenvectors of \(A\) satisfy the conditions for balancing transformation \(T\) in section \ref{bt}.
Since the eigenvalues are all negative and real the order \(\lambda{11} \geq \lambda_{22} \geq ...  \lambda_{nn}\) applies.
This enables the mode truncation described in section \ref{bt}
\begin{gather}
\tilde{x} = \begin{bmatrix}
\zeta_1 \\
\vdots \\
\zeta_r
\end{bmatrix} \quad 
\tilde{z} = \begin{bmatrix}
\zeta_{r+1} \\
\vdots \\
\zeta_n
\end{bmatrix} \quad
z = \begin{bmatrix}
\tilde{x} \\
\tilde{z}
\end{bmatrix}\\
X= \begin{bmatrix}
X_r && X_{n-r}
\end{bmatrix} \quad
X^{-1} = \begin{bmatrix}
X_r^{*} \\
X_{n-r}
\end{bmatrix} \\
\frac{d\tilde{x}}{dt} = \Delta_r \tilde{x} + X^{-1}_ru \\
y = X_r\tilde{x} + Du \,.
\end{gather}
The transfer function of that system is
\begin{gather}
G(s) = X_r(sI - \Delta_r^{-1})X_r^{T}+D = X_r diag(\frac{1}{s-\lambda_1}, ..., \frac{1}{s-\lambda_r})X_r^{T}+D \\
= \sum_{i=1}^{r} \frac{1}{s-\lambda_i} + D \,.
\end{gather}
This resembles the DCNF of that system where the terms corresponding to the \(n-r\) smallest eigenvalues are truncated.
As shown in section \ref{mtht}, this is the optimal modal truncation.
Therefore modal truncation and balanced truncation yield the same system.




