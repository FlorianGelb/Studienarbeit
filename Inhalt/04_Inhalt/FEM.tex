\section{Finite Element Method}
The finite element method (FEM) is a method to approximate solutions for differential equations (DE) within a certain domain \(\Omega\).
Assume that a DE is given by:
\begin{gather}
m, n \in \mathbb{N} \quad \zeta \in \Omega \subset \mathbb{R} \quad m \geq 1 \\
\frac{\partial^{m} y}{\partial \zeta^{m}} -  g(y) = r(\zeta, t) \label{eq-pde-gen} 
\end{gather}
It is assumed that \(g\) is a linear function that can also contain partial derivatives of \(y\) w.r.t. time, \(y\) takes the value 0 at the boundary \(\Gamma\) and \(y(\zeta, 0) = f(\zeta)\).
An approximate solution to \(y\) is given by \(\mu\), which is expressed as a sum of basis functions contained in the set \(\phi\):
\begin{gather}
\mu(\zeta, t) = \sum_{j = 1}^{N} c_{j}(t)\phi_{j}(\zeta) \label{eq-aprox}
\end{gather}
The residual is defined as:
\begin{gather}
\mathfrak{r} = \frac{\partial^{m} \mu}{\partial \zeta^{m}} -  g(\mu) - r(\zeta, t) 
\end{gather}
Furthermore the residual is required to orthogonal to all basis functions:
\begin{gather}
\langle \mathfrak{r}, \phi_{k} \rangle = 0 \quad \forall \phi_{k} \in \phi \label{eq-req-orth}
\end{gather}

Since the functions in \(\phi\) are known, it is only required to find the coefficients \(c_{j}(t)\) in \ref{eq-aprox}.
To find those coefficients \ref{eq-req-orth} needs to be expressed as follows:
\begin{gather}
\int_{\Omega} \frac{\partial^{m} \mu}{\partial \zeta^{m}} \phi_{k} \, d\zeta  -  \int_{\Omega} g(\mu) \phi_{k}        \, d\zeta = \int_{\Omega}  r(\zeta, t) \phi_{k}        \, d\zeta \quad \forall \phi_{k} \in \phi \label{eq-req-orth}
\end{gather}
If \(\mu\) is substituted with \ref{eq-aprox} the following is obtained:
\begin{gather}
\sum_{j = 1}^{N} (\int_{\Omega} \frac{\partial^{m} \phi_{j}}{\partial \zeta^{m}} \phi_{k} \, d\zeta - g(\int_{\Omega} \phi_k \phi_j)) c_{j}(t) = \int_{\Omega}  r(\zeta, t) \phi_{k}        \, d\zeta \quad \forall \phi_{k} \in \phi \label{eq-req-orth}
\end{gather}

It is also necessary to apply integration by parts to the first integral term taking into account that \(y\) at \(\Gamma\) is 0:
\begin{gather}
\int_{\Omega} \frac{\partial^{m} \phi_{j}}{\partial \zeta^{m}} \phi_{k} \, d\zeta = - \int_{\Omega} \frac{\partial^{m-1} \phi_{j}}{\partial \zeta} \frac{\partial \phi_{k}}{\partial \zeta} \, d\zeta \quad \forall \phi_{k} \in \phi \label{eq-req-orth}
\end{gather}
This formulation leads to a system of ODEs or a system of linear equations that can be solved either analytically or numerically. 

This formulation of FEM can be applied to \ref{eq-1d-h}:
\begin{gather}
\Omega = \chi \quad \Gamma = \{x_{0}, x_{n}\} \\
y(\zeta, t) = u(x, t) \quad g(u) = \frac{1}{\alpha} \frac{\partial u}{\partial t} \\
r(\zeta, t) = - \frac{1}{\alpha} h(x,t)
\end{gather}
The set of basis functions is defined as a set of piecewise linear functions.

