% ------------------------------------------------------------
% LaTeX Template für die DHBW zum Schnellstart!
% Original: https://github.wdf.sap.corp/vtgermany/LaTeX-Template-DHBW
% ------------------------------------------------------------
% ---- Präambel mit Angaben zum Dokument
\input{content/00_latex/praeambel}

% ---- Elektronische Version oder Gedruckte Version?
% ---- Unterschied: Die elektronische Version enthält keinen Platzhalter für die Unterschrift
\usepackage{ifthen}
\newboolean{e-Abgabe}
\setboolean{e-Abgabe}{false}    % false=gedruckte Fassung

% ---- Persönlichen Daten:
\newcommand{\titel}{Comparison of selected model order reduction methods}
\newcommand{\titelheader}{Comparison of selected model order reduction methods}
\newcommand{\arbeit}{Studienarbeit (T3\_3101)}
\newcommand{\studiengang}{Angewandte Informatik}
\newcommand{\studienjahr}{2020}
\newcommand{\autor}{Florian Braun}
\newcommand{\autorReverse}{Nachname, Vorname}
\newcommand{\verfassungsort}{Karlsruhe}
\newcommand{\matrikelnr}{7433149}
\newcommand{\kurs}{TINF20B1}
\newcommand{\bearbeitungsmonat}{Januar 2018}
\newcommand{\abgabe}{22. Mai 2023}
\newcommand{\bearbeitungszeitraum}{14.10.2022 - 22.05.2023}
\newcommand{\betreuerDhbw}{Lutz Gröll}

\input{content/00_latex/kopfundFusszeile}

% ---- Hilfreiches
\newcommand{\zB}{z.\,B. }   % "z.B." mit kleinem Leeraum dazwischen (ohne wäre nicht korrekt)
\newcommand{\dash}{d.\,h. }

\newcommand{\code}[1]{\texttt{#1}} % Ist einfacher zu schreiben als ständig \texttt und erlaubt
                                   % Änderungen im Nachhinein, wenn man z.B. Inline-Code anders stylen möchte.

% ---- Silbentrennung (falls LaTeX defaults falsch / nicht gewünscht sind)
\hyphenation{HANA}         % anstatt HA-NA
\hyphenation{Graph-Script} % anstatt GraphS-cript

% ---- Beginn des Dokuments
\begin{document}
\setlength{\parindent}{0pt}              % Keine Paragraphen Einrückung.
                                         % Dafür haben wir den Abstand zwischen den Paragraphen.
\setcounter{secnumdepth}{2}              % Nummerierungstiefe fürs Inhaltsverzeichnis
\setcounter{tocdepth}{1}                 % Tiefe des Inhaltsverzeichnisses. Ggf. so anpassen,
                                         % dass das Verzeichnis auf eine Seite passt.
\sffamily                                % Serifenlose Schrift verwenden.

% ---- Vorspann
% ------ Titelseite
\singlespacing
\include{content/01_standard/titelseite}  % Titelseite
\newcounter{savepage}
\pagenumbering{Roman}                    % Römische Seitenzahlen
\onehalfspacing

% ------ Erklärung, Sperrvermerk, Abstact
\include{content/01_standard/erklaerung}
%%\include{content/01_standard/sperrvermerk}
%%\include{content/02_abstract/abstract-en}
\renewcommand{\abstractname}{Abstract} % Veränderter Name für das Abstract
\begin{abstract}
\begin{addmargin}[1.5cm]{1.5cm}        % Erhöhte Ränder, für Abstract Look
\thispagestyle{plain}                  % Seitenzahl auf der Abstract Seite
This paper presents and compares four different model order reduction methods.
These are proper orthogonal decomposition, balanced truncation, modal truncation, and Hankel norm approximation.
These methods are applied to the 1D heat equation.
The heat equation is modified so that a heating process can be represented.
This equation is discretized using the finite element method to obtain a state space representation of the system.
The model order reduction methods are compared with respect to the \(L2\)- and \(H_{\infty}\) Fehelr and the runtime.

\vspace{0.25cm}



\end{addmargin}
\end{abstract}

% ------ Inhaltsverzeichnis
\singlespacing
\tableofcontents

% ------ Verzeichnisse
\renewcommand*{\chapterpagestyle}{plain}
\pagestyle{plain}
\chapter*{List of abbreviations}
\addcontentsline{toc}{chapter}{List of abbreviations} % Hinzufügen zum Inhaltsverzeichnis 

\begin{acronym}[Inverse Discrete Fourier Transform] % längstes Kürzel wird verw. für den Abstand zw. Kürzel u. Text
	\acro{BT}{Balanced Truncation}
	\acro{DCNF}{Diagonal Canonical From}
	\acro{DE}{Differential Equation}
	\acro{DFT}{Discrete Fourier Transform}
	\acro{FEM}{Finite Element Method}
	\acro{FT}{Fourier Transform}	
	\acro{FFT}{Fast Fourier Transform}
	\acro{HNA}{Hankel Norm Approximation}
	\acro{HSVD}{Hankel Singular Decomposition}
	\acro{IDFT}{Inverse Discrete Fourier Transform}
	\acro{IFFT}{Inverse Fast Fourier Transform}
	\acro{LS}{Least Squares}
	\acro{LTI}{Linear Time Invariant}
	\acro{MIMO}{Multiple Input Multiple Output}
	\acro{MOR}{Model Order Reduction}
	\acro{MT}{Modal Truncation}
	\acro{ODE}{Ordinary Differential Equation}
	\acro{PDE}{Partial Differential Equation}
	\acro{POD}{Proper Orthogonal Decomposition}
	\acro{ROM}{Reduced Order Model}
	\acro{SVD}{Singular Value Decomposition}

	
\end{acronym}

\listoffigures                          % Erzeugen des Abbildungsverzeichnisses 
\listoftables                           % Erzeugen des Tabellenverzeichnisses
\renewcommand{\lstlistlistingname}{source-codeverzeichnis}
\lstlistoflistings                      % Erzeugen des Listenverzeichnisses
\setcounter{savepage}{\value{page}}


% ---- Inhalt der Arbeit
\cleardoublepage
\pagenumbering{arabic}                  % Arabische Seitenzahlen für den Hauptteil
\chapter{Fundamentals}
This chapter covers the fundamentals necessary for understanding the model order reduction methods. First of all, a system to apply those methods is needed. Here the 1D heat equation was chosen since it is fairly easy to solve and to understand the results. After that, the method of finite elements is introduced. This method discretizes the 1D heat equation into a solvable system of ODEs since it can handle boundary conditions and initial values. One of the most important fundamentals of this paper is the singular value decomposition since it takes heavy use in the model order reduction methods covered. The last Section of this chapter will cover some control theory fundamentals. Those are needed since three model order reduction methods appear in a control theory context. Hence some explanation of the terminology is needed.
\section{Heat Equation} \label{HE}
The following partial differential equation (PDE) describes the conduction of heat  within a medium
\begin{gather}
\frac{\partial u}{\partial t} = \alpha \nabla^{2} u\,. \label{heat}
\end{gather}
Here \(u\) is a function of space and time, and \(\alpha\) is a positive constant.
For this paper \(u\) will be defined in terms of one spatial dimension \cite{Gustafsson2011}
\begin{gather}
u := u(x, t) \\
\frac{\partial u}{\partial t} = \alpha \frac{\partial^{2} u}{\partial x^{2}} \\
x \in \chi \subset \mathbb{R} \quad t \in \tau \subset \mathbb{R} \\
x_{0} \leq x \leq x_{n} \quad t_{0} \leq t \leq t_{n}\,.
\end{gather}

In order to model the conduction of heat within a medium but also a heating process, a new function \(h: \chi \times \tau \rightarrow  \mathbb{R}\) is introduced:
\begin{gather}
\frac{\partial u}{\partial t} = \alpha \frac{\partial^{2} u}{\partial x^{2}} + h(x,t)\,. \label{eq-1d-h}
\end{gather}
For this paper, the assumption is made that the initial condition is known
\begin{gather}
f: \chi \rightarrow \mathbb{R} \\
u(x, t_{0}) = f(x)\,.
\end{gather} 

Applying the Fourier transform (FT) w.r.t \(x\) to (\ref{eq-1d-h}) yields the inhomogeneous ordinary differential equation (ODE)
\begin{gather}
\hat{u} = \mathfrak{F}(u) \quad \hat{h} = \mathfrak{F}(h) \\
\frac{d}{dt} \hat{u} = -\alpha\omega^{2}\hat{u} + \hat{h}\,. \label{eq-1d-h-ft}
\end{gather}

A solution to (\ref{eq-1d-h-ft}) is given by
\begin{gather}
\hat{u} = \hat{u}_{0} + \hat{u}_{p}\,.
\end{gather}
Where \(\hat{u}_{0}\) is the homogeneous solution and \(\hat{u}_{p}\) is the particular integral.
In order to solve this ODE, the particular integral \(\hat{h}\) has to be known \cite{Papula2015}.
The choice of \(h\) is, except for some restrictions, arbitrary.
Therefore an approximate solution to \ref{eq-1d-h-ft} \(\hat{u}_{a}\) is obtained by the forward Euler scheme
\begin{gather}
\frac{d}{dt} \hat{u} \approx \frac{\Delta \hat{u}}{\Delta t} \\
\hat{u}_{t+1} = \hat{u}_{t} + \Delta t (-\alpha\omega^{2}\hat{u} + \hat{h}) \label{eq-1d-h-es} \\
\hat{u}_{a} = \begin{bmatrix}
\hat{u}_{t_{0}} & \hdots & \hat{u}_{t_{n}}
\end{bmatrix}\,.
\end{gather}
The Euler scheme is applicable if an initial condition \(\hat{u}_{0}\) is known. 
This initial condition is obtained by applying the discrete Fourier transform (DFT) to an initial temperature distribution along \(x\) \cite{Gustafsson2011b}
\begin{gather}
\hat{u}_{0} = \mathfrak{F} \{f(x)\}\,.
\end{gather} 


The forward Euler scheme is used because it is easy to implement.
Applying the inverse discrete Fourier transform (IDFT) to \(\hat{u}_{a}\), an approximate solution to (\ref{eq-1d-h}) can be obtained.
Using fast Fourier transform (FFT) and inverse fast Fourier transform (IFFT)  instead of the DFT, and IDFT reduces the processing time.
This method for solving the heat equation raises the problem that it cannot handle boundary conditions. 
Using the finite element method, the heat equation is solvable as an initial boundary value problem (IBVP).

\section{Finite Element Method}
The finite element method (FEM) is a method to approximate solutions for differential equations (DE) within a certain domain \(\Omega\).
Assume that a DE is given by:
\begin{gather}
m, n \in \mathbb{N} \quad \zeta \in \Omega \subset \mathbb{R} \quad m \geq 1 \\
\frac{\partial^{m} y}{\partial \zeta^{m}} -  g(y) = r(\zeta, t) \label{eq-pde-gen} 
\end{gather}
It is assumed that \(g\) is a linear function that can also contain partial derivatives of \(y\) w.r.t. time, \(y\) takes the value 0 at the boundary \(\Gamma\) and \(y(\zeta, 0) = f(\zeta)\).
An approximate solution to \(y\) is given by \(\mu\), which is expressed as a sum of basis functions contained in the set \(\phi\):
\begin{gather}
\mu(\zeta, t) = \sum_{j = 1}^{N} c_{j}(t)\phi_{j}(\zeta) \label{eq-aprox}
\end{gather}
The residual is defined as:
\begin{gather}
\mathfrak{r} = \frac{\partial^{m} \mu}{\partial \zeta^{m}} -  g(\mu) - r(\zeta, t) 
\end{gather}
Furthermore the residual is required to be orthogonal to all basis functions:
\begin{gather}
\langle \mathfrak{r}, \phi_{k} \rangle = 0 \quad \forall \phi_{k} \in \phi \label{eq-req-orth}
\end{gather}

Since the functions in \(\phi\) are known, it is only required to find the coefficients \(c_{j}(t)\) in \ref{eq-aprox}.
To find those coefficients \ref{eq-req-orth} needs to be expressed as follows:
\begin{gather}
\int_{\Omega} \frac{\partial^{m} \mu}{\partial \zeta^{m}} \phi_{k} \, d\zeta  -  \int_{\Omega} g(\mu) \phi_{k}        \, d\zeta = \int_{\Omega}  r(\zeta, t) \phi_{k}        \, d\zeta \quad \forall \phi_{k} \in \phi 
\end{gather}
If \(\mu\) is substituted with \ref{eq-aprox} the following is obtained:
\begin{gather}
\sum_{j = 1}^{N} ((\int_{\Omega} \frac{\partial^{m} \phi_{j}}{\partial \zeta^{m}} \phi_{k} \, d\zeta) c_{j}(t) - g((\int_{\Omega} \phi_k \phi_j d\zeta) c_{j}(t)))  = \int_{\Omega}  r(\zeta, t) \phi_{k}        \, d\zeta \quad \forall \phi_{k} \in \phi \label{eq-al}
\end{gather}

It is also necessary to apply divergence theorem to the first integral term taking into account that \(y\) at \(\Gamma\) is 0.
Since \(\zeta\) is one dimensional, the divergence theorem becomes integration by parts:
\begin{gather}
\int_{\Omega} \frac{\partial^{m} \phi_{j}}{\partial \zeta^{m}} \phi_{k} \, d\zeta = - \int_{\Omega} \frac{\partial^{m-1} \phi_{j}}{\partial \zeta^{m-1}} \frac{\partial \phi_{k}}{\partial \zeta} \, d\zeta \quad \forall \phi_{k} \in \phi \label{eq-ibp}
\end{gather}
Combining \ref{eq-al} and \ref{eq-ibp} yields:
\begin{gather}
-\sum_{j = 1}^{N} ((\int_{\Omega} \frac{\partial^{m-1} \phi_{j}}{\partial \zeta^{m-1}} \frac{\partial \phi_{k}}{\partial \zeta} \, d\zeta) c_{j}(t) + g((\int_{\Omega} \phi_k \phi_j d\zeta) c_{j}(t)))  = \int_{\Omega}  r(\zeta, t) \phi_{k}        \, d\zeta \quad \forall \phi_{k} \in \phi \label{eq-fem}
\end{gather}

This formulation leads to a system of ODEs or a system of linear equations that can be solved either analytically or numerically. 
This formulation of FEM can be applied to \ref{eq-1d-h}:
\begin{gather}
\Omega = \chi \quad \Gamma = \{x_{0}, x_{n}\} \\
y(\zeta, t) = -u(x, t) \quad g(u) = -\frac{1}{\alpha} \frac{\partial u}{\partial t} \\
m = 2 \quad r(\zeta, t) = \frac{1}{\alpha} h(x,t) \\
u(x, 0) = f(x) \quad u(x_{0}, t) = 0 \quad u(x_{n}, t) = 0
\end{gather}
The set of basis functions is defined as a set of piecewise linear functions with constant step size \(\Delta x\):
\begin{gather}
    \phi_j(x)= 
\begin{cases}
    (x - x_{j-1}) / \Delta x), \quad x_{j-1} \leq x <  x_{j}\\
    (x_{j+1} - x) / \Delta x), \quad x_{j} \leq x <  x_{j + 1}\\
    0,              \quad \text{otherwise}
\end{cases}
\end{gather}
\cite{Gustafsson2011d}
The stepsize \(\Delta x\) is defined by \(\Delta x = \frac{x_n - x_0}{n-1}\). \label{def-delta-x}
This results in the following system of ODEs:
\begin{gather}
\sum_{j=1}^N(\int_{\chi} \phi_{j}\phi_{k}dx)\frac{dc_{j}}{dt} = \alpha \sum_{j = 1}^N(-\int_{\chi} \frac{d\phi_{j}}{dx}\frac{d\phi_{j}}{dx}dx)c_{j}(t) + \int_{\chi}h(x, t) \phi_{k} dx
\quad \forall \phi_{k} \in \phi \label{eq-heat-fem}
\end{gather}
Using matrix notation this becomes:
\begin{gather}
M^{N \times N}, K^{N \times N} \\
M\dot{c} = Kc + d \label{eq-heat-almost-ss}
\end{gather}
The matrices \(M\) and \(K\) can be easily computed (Appendix \ref{ap-mat-der}):
\begin{gather}
m_{ij} = \begin{cases}
\frac{2\Delta x}{3}, \quad k = j \\
\frac{\Delta x}{6}, \quad |k - j| = 1 \\
0, \quad otherwise 
\end{cases} \label{def-mat-a}
\quad
k_{ij} = \begin{cases}
\frac{-2\alpha}{\Delta x}, \quad k = j \\
\frac{\alpha}{\Delta x}, \quad |k - j| = 1 \\
0, otherwise
\end{cases}
\end{gather}
However it is necessary to approximate \(d\) for each point in time using numerical integration schemes.
Furthermore to solve this system of ODEs numerically an initial condition \(c_{0}\) has to be known.
\cite{Gustafsson2011b}
It can be obtained using a least squares (LS) approach:
\begin{gather}
\sum_{j = 1}^N \langle \phi_j, \phi_k \rangle c_{j}(0) = \langle f, \phi_{k} \rangle \quad \forall \phi_k \in \phi \\
Mc_{0} = F \\
c_{0} = M^{-1}F
\end{gather}
\cite{Gustafsson2011c}
Observe that by multiplying \ref{eq-heat-almost-ss} with \(M^{-1}\) (\ref{ap-K-inv}) yields a system of ODEs:
\begin{gather}
\dot{c} = M^{-1}Kc + M^{-1}d
\end{gather}
This system of ODEs can be solved using an euler scheme:
\begin{gather}
c_{t+1} = dt M^{-1}(Kc + d) + c_{t}
\end{gather}
However keep in mind that vector \(d\) is time dependent and has to be recomputed for each time step.
Using \ref{eq-aprox} and the computed coefficients \(c\) the function \(u(x,t)\) can be approximated.
\section{Singular Value Decomposition}
The Singular Value Decomposition (SVD) is a matrix factorization with guaranteed existance.
It can be used to obtain low rank approximations of a matrix or pseudo inverses for ill posed linear system of equations.
It is also related to FT by providing a data specific set of orthogonal bases instead of a gerneric set of sines and cosines. For this paper the SVD will be used for generating low rank approximations of matrices.
\cite{brunton_kutz_2019}
\subsection{Properties}
A matrix \(X \in \mathbb{C}^{n \times m}\) can be decomposed in the following way:
\begin{gather}
X = U \Sigma V^{*}
\end{gather}
Here \(U \in \mathbb{C}^{n \times n}\) and \(V \in \mathbb{C}^{m \times m}\) are unitary matrices and \(\Sigma \in \mathbb{R}^{n \times m}\) is a real valued oredered diagonal matrix.
The columns of \(U\) provides a set of orthonormal basis vectors for the column space of \(X\), \(V\) contains orthonormal basis vectors for the row space of \(X\). The matrix \(X\) asigns a magnitude ('importance') to the product of \(U\) and \(V^{*}\)  \cite{brunton_kutz_2019}.
Since \(U\) and \(V\) are unitary they have the following property:
\begin{gather}
U^{*}U = UU^{*} = I \\
V^{*}V = VV^{*} = I
\end{gather}
\cite{SZABO2015385}


In case \(n \geq m\) the so called economy SVD can be used to factorize the matrix \(X\):
\begin{gather}
X = \begin{bmatrix}
\hat{U} & \hat{U}^{\bot}
\end{bmatrix} 
\begin{bmatrix}
\hat{\Sigma} \\
0
\end{bmatrix}
V^{*} = \hat{U} \hat{\Sigma} V^{*}
\end{gather} 
The economy SVD omits rows only containing zeros in \(\Sigma\) and the according columns of \(U\).
Therefore the dimensionality of \(\hat{U}\) and \(\hat{\Sigma}\) is less or equal to the dimensionality of \(U\) and \(\Sigma\).
 \cite{brunton_kutz_2019}

\subsection{Low-rank approximation}
A usefull property of the SVD is that it can be used to find an hierachy of rank-\(r\) approximation for a given matrix \(X\).
An matrix \(\tilde{X}\) that approximates \(X\) is obtained by:
\begin{gather}
\tilde{X} = arg\,min ||X - \tilde{X}||_F = \tilde{U}\tilde{\Sigma}\tilde{V}^{*}	\\
s.t. rank(\tilde{X}) = r
\end{gather}	
Here  \(\tilde{U}\) and \(\tilde{V}\) denote matrices obtained takeing the first \(r\) columns of \(U\) and \(V\). The matrix \(\tilde{\Sigma}\) is a \(r \times r\) sub-block of \(\Sigma\).
This is alsow known as the Eckard-Young theorem.
\cite{brunton_kutz_2019}





\section{Fundamentals of Control Theory}

\subsection{Problems in Control Theory}
There are mainly three different problems in control theory.
To analyse a system it has to be identified first.
A mathematical model has to be found that relates the input of that system to it's output.
After a model has been found, it is helpfull to simulate the system to find out which input results in which output.
Simulations have to fit the system and offer the advantage over real experiments of reducing cost and risks attached to them.
The last of the three major problems is the control problem.
Now that the underlying model is known and the system can be simulated, a controler for that system can be designed.
Here the controler is a system that feeds some input to the system that is to be controlled to generate some desired output.
\cite{DouglasBa}
For this paper the second problem is the most important one, since the goal of this paper is to compare some methods to make simulations of a system more efficient.
\subsection{LTI Systems}
A linear and time invariant (LTI) system is a kind of system that fulfills the following conditions:
\paragraph{Homogenity:}
For a system to be a LTI system, the condition that the output of a system \(y(t)\) relates to the input \(u(t)\) by a linear operator \(h\):
\begin{gather}
y(t) = h(u(t)) \\
cy(t) = h(cu(t))
\end{gather}
This means that if the input \(u\) changes in magnitude by a constant factor of \(c\) the output \(y\) also changes by the factor of \(c\).
\paragraph{Superposition:}
The second requirement is superpostion:
\begin{gather}
y_1(t) + y_2(t) = h(u_1(t) + u_2(t))
\end{gather}
The sum of two outputs \(y_1\) and \(y_2\) has to equal the output of the system with the sum of the inputs \(u_1\) and \(u_2\) as input.
\paragraph{Time Invariance:}
A system that is time invariant has the property that if the input \(u\) is shifted in time by some constant \(\tau\) the resulting output is also shifted in time by the same constant:
\begin{gather}
y(t - \tau) = h(u(t-\tau))
\end{gather}
LTI systems are important because they can be used to approximate non LTI systems over some region.
This is usefull since LTI systems are well understood.
\cite{DouglasB}
\subsection{State Space representation}
A LTI system can be expressed as a system of ODEs that relates the input \(u\) to its output \(y\).
The vector \(x\) denotes the states of a system.
\begin{gather}
\dot{x} = Ax + Bu \quad x(t_0) = x_0\\
y = Cx + Du
\end{gather}
The matrices \(A \in \mathbb{R}^{n \times n}\),
\(B \in \mathbb{R}^{n \times m}\),
\(C \in \mathbb{R}^{q \times n}\) and
\(D \in \mathbb{R}^{q \times n}\) are constant matrices.
\cite{BennerGrivet}

\subsection{Transfer Function}
\subsection{Controallability and Observability}


\chapter{Model Order Reduction}
\section{Introduction}
Model Order Reduction (MOR) is a technique to reduce the computational effort of simulate a system using mathematical models.
This is done by modeling only the dominant behaviours of those systems \cite{+2021}.
In control theory a system is described by a system of ODEs:
\begin{gather}
\frac{d}{dt} x = Ax + Bu \\
y = Cx + Du
\end{gather}
This is also called State Space (SS).
The State Space 



\section{Proper Orthogonal Decomposition}
The proper orthogonal decomposition (POD) is a method for model order reduction.
The reduction in computational effort is done by approximating a solution to a PDE using an orthogonal expansion.
The basis functions are obtained by decomposing a set of solutions to othe PDE using the SVD.
\subsection{Orthogonal Expansion}
A function \(f: \mathbb{R} \times \mathbb{R} \rightarrow \mathbb{R}\) can be represented by the series:
\begin{gather}
f(x, t) = \sum_{i = 1}^{\infty}a_i(t)\phi_i(x) \\
x, t \in \mathbb{R} \label{ref-orth-exp}
\end{gather}
Here all \(\phi(x)\) are orthorgonal basis functions.
\begin{gather}
\langle\phi_i, \phi_j\rangle =\begin{cases}
1, \quad \text{if } i = j \\
0, \quad \text{else}
\end{cases} \label{phi-orth}
\end{gather}
By using a finite sum instead of the entire  series \(f\) can be approximated:
\begin{gather}
\tilde{f}(x, t) = \sum_{i = 1}^{n}a_{i}(t)\phi_{i}(x) \label{ref-orth-aprox}
\end{gather}
Since all \(\phi\) are known, \ref{ref-orth-aprox} has to be solved for the set \(\{a_0, ..., a_n\}\).
In case \(f\) is known, the solutions contained in this set can be calculated as:
\begin{gather}
a_i(t) = \frac{\langle \phi_i, f \rangle}{\langle \phi_i, \phi_i \rangle} \label{sol-ai}
\end{gather}
\cite{Gustafsson2011e}
\subsection{Proper Orthogonal Decomposition for PDEs}
Since a PDEs are equations in terms of partial derivatives, the notation \(\mathscr{P}(\partial x) u\) is introduced, which denotes a differential operator in terms of the spacial variables \(x\) for a function \(u(x,t,p\).
The vector \(p\) contains some parameters.
This is done to provide a more abstract way to denote PDEs:
\begin{gather}
\frac{\partial u}{\partial t} = \mathscr{P}(\partial x) u
\end{gather}
\cite{Gustafsson2011f}
Solving for \(u\) is often dificult to imposible.
A method that is often used to solve PDEs is called seperation of variables.
This spereation of variables assumes, that the underlying solution \(u(x, t)\) can be expressed by \ref{ref-orth-exp}, to solve for \(a_k, 0 \leq k \leq n\).
Since it is not practical to compute an infite series, \(u\) only gets appriximated by using \ref{ref-orth-aprox} instead:
\begin{gather}
\sum_{i = 1} ^{n} \frac{\partial a_i}{\partial t} \phi_i = \sum_{i = 1} ^{n} \mathscr{P}(\partial x) \phi_i a_i \label{label-u-aprox} 
\end{gather}
By discretizing the spacial dimension got along \(x\) \ref{label-u-aprox} can be expressed using marix notation:
\begin{gather}
\Phi = \begin{bmatrix}
\phi_0, ..., \phi_n
\end{bmatrix} \label{mat-phi}\\
\Phi \frac{d}{dt}a = \mathscr{P}(\partial x) \Phi a
\end{gather}
Since the solution \(u\) is unknown \ref{sol-ai} cannot be computed.
However the fact, that all \(\phi\) are orthogonal to each oter, can be used to solve for all \(a_k\).
It can be done by computing the inner product of \ref{label-u-aprox} with all basis funktions:
\begin{gather}
\langle \sum_{i = 1} ^{n} \frac{\partial a_i}{\partial t} \phi_i, \phi_k \rangle = \langle\sum_{i = 1} ^{n} \mathscr{P}(\partial x) \phi_i a_i, \phi_k \rangle \quad \forall 0 \leq k \leq n \label{u-galer}
\end{gather}
This resembles the Galerking projection.
In matrix notation it can be expressed:
\begin{gather}
\Phi^{*}\Phi \frac{d}{dt}a = \Phi^{*}\mathscr{P}(\partial x) \Phi a
\end{gather}
By considering \ref{phi-orth}  the equiation \ref{u-galer} can be expressed as a system of ODEs which can be solved:
\begin{gather}
\frac{d}{dt} a = \Phi^{*}\mathscr{P}(\partial x) \Phi a
\end{gather}
After the vector of coefficiants \(a\) for each time step has been computed, the solution can be assembled:
\begin{gather}
u(x, t) \approx \Phi(x)a(t) \label{u-aprox-pod}
\end{gather} 
\cite{brunton_kutz_2019c}
\subsection{Choosing basis vectors}
As discussed in the previous section, a set of basis vectors can be used to generate approximate solutions to PDEs.
However it was not discussed how those basis functions are chosen.
For POD to work, a so called snapshot matrix \(X\) has to be available.
This snapshot matrix stores a set of solutions where \(x_k\) is the solution for a PDE at time step \(k\Delta t\):
\begin{gather}
X = \begin{bmatrix}
x_1, ..., x_m
\end{bmatrix}
\end{gather}
The solutions can be obtained by conducting an expirment on a physical system that is described by the PDE or by simulating the evolution of that PDE.
In this paper the solution is obtained using FEM \ref{FEM}.
SVD is used to decompose the snapshot matrix \(X\).
Since the columns of the snapshot matrix contain spacial information at a given point in time and the basis vectors are supposed to encode the spacial information of a solution \(u\) the \(r\) most dominant left singular values have to be extracted:
\begin{gather}
\tilde{X} = \tilde{U}\tilde{\Sigma}\tilde{V}^{*} \\
\Phi = [u_1, ..., u_r] \label{PHI}
\end{gather}
Here the FEM solution is obtained by discretizing the PDE into a large number(\(n\)) of spacial nodes.
This results in a high dimensional system of ODEs.
Since \(\Phi\) contains only \(r\) column (\(r << n\)) vectors a reduction in order can be achieved.
However the benefits of this reduced order model are only relevant after the PDE has been solved once using a high dimensional system of ODEs.
\cite{brunton_kutz_2019c}
\subsection{ROM for heat equation}
In order to apply pod to heat equation a snapshot matrix \(X\) has to be generated.
This is done by the FEM solver described in \ref{FEM}.
After that \(X\) gets decomposed using the SVD.
The modes contained in \(\Phi\) is obtained by truncating the left singular vectors of \(X\) according to \ref{PHI}.
Substituting \(\mathscr{P}\) in \label{label-u-aprox} with heat equation \ref{eq-1d-h} results in the following:
\begin{gather}
\Phi \frac{\partial a}{\partial t}  = \alpha \frac{\partial^{2} \Phi}{\partial x^{2}} a + h\\
\frac{\partial a}{\partial t} = \alpha \Phi^{*}  \frac{\partial^{2} \Phi}{\partial x^{2}} a + \Phi^{*}h
\end{gather}
\newpage
This system of ODEs can now be solved using a Runge-Kutta scheme.
Note that \(\Phi\) the contains numeric values.
Therefore derivatives are unstable, especially at the first and last entries of the column vectors of \(\Phi\).
To reduce this problem the method used to compute the second derivative of \(\Phi\) is the so called spectral derivative.
The discrete spectral derivative works by computing the FFT of a vector.
That vector gets multiplied by \((ik)^{d}\), \(d\) is the order of the derivative and \(k\) are the discrete wave numbers.
After that step the IFFT is applied to obtain the derivative of the original vector.
\begin{gather}
f \in \mathbb{C}^{n}, \quad k = [-\frac{n}{2} \hdots \frac{n}{2}]^{T} \\
\frac{df}{dx} = \mathfrak{F}^{-1}\{i \frac{2 \pi k}{n} \mathfrak{F}\{f\}\} \\
\frac{d^{2}f}{dx^{2}} = \mathfrak{F}^{-1}\{-\frac{2 \pi k}{n} \mathfrak{F}\{f\}\} \\
\end{gather}
\cite{brunton_kutz_2019f}



\section{Balanced Truncation}
Balanced truncation is a method for model order reduction.
The goal of balanced truncation is to approximate a system using only the most relevant modes of the system.
The difference to POD is that the modes are not selected by the variance they capture but by the controllability and observability of the modes.
This is done by finding a coordinate transform.
\subsection{Balanceing Coordinate Transform}
To find a reduced order model using balanced truncation a orthonormal coordinate transform is applied: \(x = Tz\).
This yields a new system:
\begin{gather}
\dot{z} = \hat{A}z + \hat{B}u \label{z1}\\
y = \hat{C}z + Du \label{z2} \\
\hat{A} = T^{-1}AT \quad \hat{B} = T^{-1}B \quad \hat{C} = CT \label{red-sys-mat}
\end{gather}
The gramians of this ROM can be obtained by applying  \ref{gram-obsv} and \ref{gram-ctrl} to \ref{red-sys-mat}.
This yields \(\hat{W}_c = T^{-1}W_cT^{-*}\) and \(\hat{W}_o = T^{*}W_oT\) with  \(T^{-*} := (T^{-1})^{*} := (T^{*})^{-1}\).
A requirement \(T\) has to satisfy is that it has to make the observability and controllability gramians of the ROM equal and diagonal.
\begin{gather}
\hat{W}_c = \hat{W}_o = \Delta \\
\hat{W}_c \hat{W}_o = \Delta^{2} \\
T^{-1}W_cW_oT = \Delta^{2} \\
W_cW_oT = T\Delta^{2} \label{eigendec}
\end{gather}
Since \(\Delta\) is a diagonal matrix, \ref{eigendec} is equal to the eigendecomposition of \(W_cW_o\).
Therefore \(T\) contains the eigenvectors of \(W_cW_o\).
However \(T\) needs to be rescaled to make \(\hat{W}_c\) and \(\hat{W}_o\) equal.
Here \(T_u\) denotes the unscaled eigenvectors of this eigendecomposition that yields gramians that are not equal to each other:
\begin{gather}
T_u^{-1}W_cT_u^{-*} = \Delta_c \label{1}\\
T_u^{*}W_cT_u = \Delta_o \label{2}
\end{gather}
Scaling \(T_u\) by some diagonal matrix \(\Delta_s\) results in \(\Delta_c = \Delta_o\):
\begin{gather}
\Delta_s = \Delta_c^{\frac{1}{4}}\Delta_o^{-\frac{1}{4}} \\
T = T_u \Delta_s
\end{gather}
Another important property of this transform is that the new coordinates are hierarchically ordered by observability and controllability.
It can be shown by deriving some unit vector \(\zeta\) that maximizes the controllability and observability:
\begin{gather}
\zeta = arg\max \, \zeta^{*}W_cW_o\zeta \quad s.t. ||\zeta||_2^{2} = 1 \\
\frac{d}{d\zeta} \zeta^{*}W_cW_o\zeta - 2\lambda \zeta = 0 \label{opt1}
\end{gather}
%https://www.matheplanet.com/matheplanet/nuke/html/viewtopic.php?rd2&topic=128338&start=0#p937673
As shown here \cite{170373} the remaining derivative can be solved in the following way:
\begin{gather}
\frac{d}{dx} x^{*}Ax = 2Ax \label{der-mat}
\end{gather}
This holds if \(A\) is symmetric. 
Here both \(W_c\) and \(W_o\) share the same set of eigenvectors \ref{1} and \ref{2}, therefore they commute \cite{170371}.
This means that the product of \(W_c\) and \(W_o\) is also symmetric \cite{170372}.
Applying \ref{der-mat} to \ref{opt1} yields:
\begin{gather}
W_cW_o\zeta = \lambda \zeta
\end{gather}
Since \(\lambda\) is the Lagrange multiplier, it is a scalar. 
It is clear that \(\zeta\) is a eigenvector.
Since the eigenvalues in \(\Delta\) contain information about how much each eigenvector gets scaled by multiplying it with \(W_cW_o\) the eigenvectors can be ordered by controllability and observability.

\subsection{Mode Truncation}
Since the goal of balanced truncation is to find a ROM of rank \(r\) that approximates the original system of rank \(n\) with \(r << n\) it is necessary to truncate the balanced system.
This yields the following system:
\begin{gather}
\frac{d\tilde{x}}{dt} = \tilde{A}\tilde{X} + \tilde{B}u \\
y = \tilde{C}\tilde{x} + \tilde{D}u
\end{gather}
The new state vector \(\tilde{x}\) is defined as:
\begin{gather}
\tilde{x} = \begin{bmatrix}
z_1 \\
\vdots \\
z_r
\end{bmatrix} \quad 
\tilde{z} = \begin{bmatrix}
z_{r+1} \\
\vdots \\
z_n
\end{bmatrix} \quad
z = \begin{bmatrix}
\tilde{x} \\
\tilde{z}
\end{bmatrix} \label{decomp-vecs}\\
T = \begin{bmatrix}
\Psi & T_t
\end{bmatrix} \quad
T^{-1} = S = \begin{bmatrix}
\Phi^{*} \\
S_t
\end{bmatrix} \label{decomp-mats}
\end{gather}
By substituting \ref{decomp-vecs} and \ref{decomp-mats} into the system in \ref{z1} and \ref{z2} the system becomes:
\begin{gather}
\frac{d}{dt} \begin{bmatrix}
\tilde{x} \\
\tilde{z}
\end{bmatrix} = \begin{bmatrix}
\Phi^{*}A\Psi & \Phi^{*}AT_t \\
S_tA\Psi & S_tAT_t
\end{bmatrix} \begin{bmatrix}
\tilde{x} \\
\tilde{z}
\end{bmatrix}
+ \begin{bmatrix}
\Phi^{*}B \\
S_tB
\end{bmatrix} u \\
y = \begin{bmatrix}
C \Psi & CT_t
\end{bmatrix} \begin{bmatrix}
\tilde{x} \\
\tilde{z}
\end{bmatrix} + Du
\end{gather}
However the only relevant part of this system is:
\begin{gather}
\frac{d\tilde{x}}{dt} = \Phi^{*}A\Psi\tilde{x} + \Phi^{*}Bu \\
y = C\Psi\tilde{x} + Du 
\end{gather}
since this is the only part necessary to calculate \(\tilde{x}\)
\cite{brunton_kutz_2019e}.

\subsection{Computing Balanced Truncation}
Since the gramians for controllability and observability are too expensive to compute for large systems the so called empirical gramians are used as an approximation.
The empirical gramians are calculated by using the discrete-time system matrices from \ref{disc-a} and \ref{disc-b}:
\begin{gather}
\mathscr{C}_d = [B_d A_dB_d \hdots dA^{m_o -1}B_d] \\
\mathscr{O}_d = \begin{bmatrix}
C_d \\
C_dA_d \\
C_dA_d^{m_o - 1}
\end{bmatrix} \\
W_c^e = \mathscr{C}_d^{*}\mathscr{C}_d \\
W_o^e = \mathscr{O}_d^{*}\mathscr{O}_d \\
m_o, m_c << rank(A)
\end{gather}
Now these empirical gramians can be used for obtaining the balancing coordinate transform \cite{brunton_kutz_2019e}.
\subsection{State Space Representation from Heat Equation} \label{heat-ss}
To apply balanced truncation to the heat equation \ref{eq-1d-h} a state space representation of that system has to be found first.
Note that the system of ODEs resulting from FEM \ref{almost-almost-ss} resembles a state space representation:
\begin{gather}
x := c \quad x_0 := c_0 \\
A := M^{-1}K \quad B:= M^{-1} 
\end{gather}
Since the system is realized as simulation all states can be accurately measured \(C := I\) and there is no feed through \(D := 0_{qm}\).
Now the already described steps for balanced truncation can be applied to the system.



\chapter{Implementation}
The implementation of the discussed methods for solving the heat equation and for model order reduction was done in Matlab.
Matlab was chosen as the programming language because it natively features matrix multiplication which finds heavy use in the previously mentioned methods.
The second reason for this selection is that there exist ToolBoxes that already implement certain model order reduction methods such as MORLAB \cite{benner_werner} or MOR toolbox \cite{MORT}.
The following figure shows the class diagram of the implementation:
\newline

\begin{figure}[H]
\includegraphics[ width=\textwidth]{images/class}
\caption{Class diagramm of MOR and FEM implementation}
\end{figure}
\section{Class main}
The class main is responsible for generating the finite element solution, model order reduction steps and plotting the results. The process can be seen in the following flow chart:
\begin{figure}[H]
\centering
\includegraphics[ width=\textwidth]{images/main-seq}
\caption{Flow chart of main class}
\end{figure}
The first step is to generate a parameter object.
The parameter object stores all parameters in order to increase the transparency and robustness of the program.
The second step is to iterate the array of stated algorithms to solve the heat equation.
The options are to solve the heat equation using finite element method \ref{FEM} or using Fourier transform \ref{HE}.
After all solver objects have been generated, the according solutions are being computed.
After that, the MOR methods are displayed.
The final step is to display the solutions.
\section{Class containerFEM}
The class containerFEM is responsible for generating a solution using finite element method.
FEM is implemented in the following way:

\begin{figure}[H]
\includegraphics[ width=\textwidth]{images/seq-fem}
\caption{Flow chart for FEM class}
\end{figure}

The first step is to set the parameters.
After that the matrices discussed in \ref{FEM} are being constructed.
The next step is to solve the resulting system of ODEs and pass the solution to a solution object.
\subsection{Construct Matrices}
As defined in \ref{def-mat-a}  the matrices \(K\) and \(M\) have to be constructed.
Also to compute the initial condition given by \(u_0\) \(F\) has to be known \ref{F}.
This is done by the following method:
\begin{algorithm}[H]
\caption{Construct matrices \(K\), \(M\) and \(F\)}
\begin{algorithmic}[1]
\State $ii \gets \frac{2}{3} \Delta nodes$
\State $ij \gets \frac{1}{6} \Delta nodes$
\State $F \gets zeros(nodes, 1)$
\State $K \gets zeros(nodes)$
\State $M \gets zeros(nodes)$
\For{$i=1  \; \textbf{to} \; nodes$}
\State $F(i) \gets trapz(\phi_i \cdot u_0)$
\EndFor 
\For{$i=1  \; \textbf{to} \; nodes$}
\State $ K(i, i) \gets \frac{-2}{\Delta nodes}$
\State $ M(i, i) \gets ii$
\If{$i-1 > 1$}
\State $ K(i, i-1) \gets \frac{1}{\Delta nodes}$
\State $ M(i, i-1) \gets ij$
\EndIf
\If{$i+1 < nodes + 1$}
\State $ K(i, i+1) \gets \frac{1}{\Delta nodes}$
\State $ M(i, i+1) \gets ij$
\EndIf
\EndFor 
\State $\textbf{return} [F, K, M]$
\end{algorithmic}
\end{algorithm}

\subsection{Compute Vector d(t)}
As discussed in \ref{FEM} vector \(d\) has to be computed for each time step:
\begin{algorithm}[H]
\caption{Construct vector d}
\begin{algorithmic}[1]
\State $d \gets zeros(nodes, 1)$
\For{$i=1  \; \textbf{to} \; nodes$}
\State $d(i) \gets trapz(\phi_i \cdot h(x, t_0))$
\EndFor 
\State $\textbf{return } d$
\end{algorithmic}
\end{algorithm}
The entries of vector \(d\) become the integral of the product  of a basis function and \(h\) evaluated at time step \(t_0\).
This time step is a argument of that method.

\subsection{Solve System of ODEs}
The most important step in the process of generating a solution is to solve the system of ordinary differential equations that FEM yields:
\begin{algorithm}[H]
\caption{Solve system of ODEs using euler scheme}
\begin{algorithmic}[1]
\State $[F, K, M] \gets \textit{construct\_matrices()}$
\State $C \gets zeros(nodes, n\_time\_steps)$
\State $c_{0} \gets M^{-1}F$
\State $C(:, 1) \gets c_{0}$
\State $M(1, :) \gets [0, \hdots, 0]$
\State $M(end, :) \gets [0, \hdots, 0]$
\State $N \gets M^{-1}K$
\For{$t=2  \; \textbf{to} \; n\_time\_steps$}
\State $d \gets \textit{generate\_h\_disc(t)}$
\State $c_{n} \gets \Delta t N c_{0} + M^{-1} h + c_{0} $
\State $c_{0} \gets c_{n}$
\State $C(:, t) \gets c_{n}$
\EndFor
\State $S \gets []$
\For{$t=1  \; \textbf{to} \; n\_time\_steps$}
\State $c \gets C(:, t)$
\State $\textit{interpol} \gets \textit{interp1(linspace(0, L, nodes),c, X)}$
\State $S(:, t) \gets \textit{interpol}$
\EndFor
\State $\textbf{return } \textit{solution(S, "FEM", 0, 0)}$
\end{algorithmic}
\end{algorithm}
In the first line the matrices \(F\), \(K\) and \(M\) are retrieved.
After that the initial vector of coefficients \(c_0\) is computed using LS \ref{c0} in line three.
The next two following lines force the boundary conditions as stated in \ref{force-bound}.
In the lines 8 to 13 \ref{fem-euler} is implemented.
In the last step the coefficients are interpolated in spatial direction to fit the given domain \(X\) and stored in a solution object.


	





\setlength{\parskip}{0.5\baselineskip}  % Abstand zwischen Absätzen
\rmfamily
\renewcommand*{\chapterpagestyle}{scrheadings}
\pagestyle{scrheadings}
\onehalfspacing

% ---- Literaturverzeichnis
\cleardoublepage
\renewcommand*{\chapterpagestyle}{plain}
\pagestyle{plain}
\pagenumbering{Roman}                   % Römische Seitenzahlen
\setcounter{page}{\numexpr\value{savepage}+1}
\printbibliography[title=Literaturverzeichnis]

% ---- Anhang
\appendix
\clearpage
\pagenumbering{Roman}  % römische Seitenzahlen für Anhang
\chapter{Appendix}
\section{Deriving Matrices for FEM Using Piecewise Linear Functions} \label{ap-mat-der}
The so called triangle function is defined as follows:
\begin{gather}
    \phi_j(x)= 
\begin{cases}
    (x - x_{j-1}) / \Delta x), \quad x_{j-1} \leq x <  x_{j}\\
    (x_{j+1} - x) / \Delta x), \quad x_{j} \leq x <  x_{j + 1}\\
    0,              \quad \text{otherwise}
\end{cases} \label{def-phi}
\end{gather}
\cite{Gustafsson2011d} \\
The following integrals have to be evaluated:
\begin{gather}
\int_{\chi} \phi_{j}\phi_{k}dx \quad \forall \phi_{k} \in \phi   \label{int-1} \\
-\int_{\chi} \frac{d\phi_{j}}{dx}\frac{d\phi_{k}}{dx}dx  \quad \forall \phi_{k} \in \phi  \label{int-2}
\end{gather}
With \(\chi \subset \mathbb{R}\). 
Note that the product of two functions \(\phi_j\) and \(\phi_k\) and their derivatives is only under two conditions not zero:

1. \(k = j\)

Considering this case the integral  \ref{int-1} becomes:
\begin{gather}
\int_{x_{j-1}}^{x_{j}} \phi_j^{2} dx + \int_{x_{j}}^{x_{j+1}} \phi_j^{2} dx 
\end{gather}
Because of symmetry only one of the above integrals have to computed:
\begin{gather}
2 \int_{x_{j-1}}^{x_{j}} \phi_j^{2} dx \\
= \frac{2}{\Delta x^2} \int_{x_{j-1}}^{x_{j}} (x-x_{j-1})^{2} dx \\
\frac{2}{3 \Delta x^{2}} \left[ (x - x_{j-1})^3\right]_{x_{j-1}}^{{x_j}} = \frac{2}{3 \Delta x^{2}} \Delta x^3 = \frac{2}{3}\Delta x
\end{gather}
Integral \ref{int-2} for \(i = j\) taking symmetry into account becomes:
\begin{gather}
-\int_{x_{j-1}}^{x_{j+1}} (\frac{d \phi_j}{dx})^2 dx = -\frac{1}{\Delta x^2} \int_{x_{j-1}}^{x_{j+1}} 1 dx\\
 = -\frac{1}{\Delta x^2}  \left[ x \right]_{x_{j-1}}^{x_{j+1}} = -\frac{2}{\Delta x}
\end{gather}

2. \(|j - k| = 1\)
\ref{int-1} becomes:
\begin{gather}
\frac{1}{\Delta x^2} \int_{x_j}^{x_{j+1}} (x-x_j)(x_{j+1} - x)dx \\
= \left[ \frac{1}{2} x^2 x_{j+1} - \frac{1}{3} x^3 - x x_{j+1} x_{j} + \frac{1}{2} x^2 x_j \right]_{x_j}^{x_{x_j+1}} 
= \frac{1}{6 \Delta x^2} \Delta x^3 = \frac{1}{6} \Delta x
\end{gather}
 
Finally \ref{int-2} has to be evaluated for this condition:
\begin{gather}
-\int_{x_j}^{x_{j+1}} \frac{d\phi_{j}}{dx}\frac{d\phi_{j+1}}{dx}dx = \frac{1}{\Delta x^2} \int_{x_j}^{x_{j+1}} 1 dx =  \frac{1}{\Delta x^2} \left[ x \right]_{x_{j}}^{x_{j+1}} = \frac{1}{\Delta x}
\end{gather}
\newpage
\section{Proof that Matrix M is Invertible}
\label{ap-K-inv}
Let \(M_{n}\) be a matrix with \(M_{n} \in \mathbb{R}^{n \times n}\) given by:
\begin{gather}
m_{ij} = \begin{cases}
a, \quad k = j \\
b, \quad |k - j| = 1 \\
0, \quad otherwise 
\end{cases}
\end{gather}
It's determinant is given by the Laplace expansion:
\begin{gather}
det(M_n) = \sum_{j=1}^{n} (-1)^{i+j} a_{ij} N_{ij} \quad \forall i
\end{gather}
\(N_{ij}\) is the determinant of the matrix \(M'\) that is obtained by removing the \(i^{th}\) row and \(j^{th}\) column of \(M_n\).
This expression can be simplified using the definition of \(m_{ij}\):
\begin{gather}
det(M_n) = a N_{11} - b N_{12} \label{1}
\end{gather}
\(N_{11}\) is equivalent to \(det(M_{n-1})\), since the indices of rows and columns of \(M'\) are in consecutive order and \(M'\) is a \(n-1 \times n-1\) matrix :
\begin{gather}
N_{11} = det(M') \\
M' = \begin{bmatrix}
m_{22} & \dots & m_{2n}\\
\vdots & \ddots & \vdots \\
m_{n2} & \dots & m_{nn}
\end{bmatrix}
\end{gather}
\(N_{12}\) can be obtained by calculating the determinant of \(M'\) using the Laplace expansion:
\begin{gather}
M' = \begin{bmatrix}
m_{21} & m_{23} & \dots & m_{2n}\\
\vdots & \vdots & \ddots & \vdots \\
m_{n1} & m_{n3} & \dots & m_{nn}
\end{bmatrix} \\
det(M') = m_{21} \cdot det(M'') \label{an1} \\
M'' = \begin{bmatrix}
m_{33} & \dots & m_{3n}\\
\vdots & \ddots & \vdots \\
m_{n3} & \dots & m_{nn}
\end{bmatrix}
\end{gather}

In \ref{an1} only the stated term has to be evaluated since all entries of the first column of the second sub matrix are zero. 
Therefore the determinant is zero.
The row and column indices of \(M''\) are in consecutive order and it is a \(n-2 \times n-2\) matrix.
Therefore \(M''\) is equivalent to \(M_{n-2}\).
\ref{1} becomes:
\begin{gather}
det(M_n) = a \cdot det(M_{n-1}) - b^{2} \cdot det(M_{n-2})
\end{gather}
Furthermore this implies \(det(M_0) = 1\):
\begin{gather}
det(M_2) = a^{2} - b^{2} =  a \cdot det(M_1) - b^{2} \cdot 1 \\
\Rightarrow det(M_0) = 1
\end{gather}
Using the definition of \ref{def-mat-a} and \ref{def-delta-x} this can be seen as the following sequence:
\begin{gather}
a_0 = 1, \; a_1 = \frac{2 \Delta x}{3} \\
a_{n+1} = \frac{2 \Delta x}{3} \cdot a_{n} - \frac{\Delta x^2}{36} \cdot a_{n-1} 
\end{gather}
As described here \cite{Michael2017} a recursive sequence converges if it is monotone and has a limit.
A proof by induction shows that this sequence is monotone for \(n \geq 1\).


Base case:
\begin{gather}
a_{2} = (\frac{2 \Delta x}{3})^{2} - \frac{\Delta x^{2}}{36} = \Delta x^{2} (\frac{4}{9} - \frac{1}{36}) < \frac{2 \Delta x}{3} = a_{1}
\end{gather}
Induction step: Assuming that \(a_k < a_{k-1}\) holds, \(a_{k+1} < a_{k}\) also holds:
\begin{gather}
a_{k+1} = \frac{2 \Delta x}{3} \cdot a_{k} - \frac{\Delta x^2}{36} \cdot a_{k-1}  < \frac{2 \Delta x}{3} \cdot a_{k-1} - \frac{\Delta x^2}{36} \cdot a_{k-2} = a_{k} 
\end{gather}
The limit of this sequence is as follow:
\begin{gather}
\alpha = \lim_{n \to \infty} a_{n+1} = \lim_{n \to \infty} \Delta x \cdot \frac{2}{3} \cdot \lim_{n \to \infty} a_{n} - \lim_{n \to \infty} \Delta x^{2} 
\cdot \frac{1}{36} \cdot \lim_{n \to \infty} a_{n-1} 
= 0 \cdot \alpha - 0 \cdot \alpha = 0
\end{gather}
Since this series is monotone and converges to zero as \(n\) goes to infinity, there is no \(n \in \mathbb{N}\) for which \(a_n = 0\). 
Therefore the determinant of the matrix \(M\) defined in \ref{def-mat-a} is not zero and \(M\) is invertible.
\section{Equivalence of Piecewise Linear Polynomials and Linear Interpolation} \label{ap-lin-interp}
A piecewise linear polynomial in the form of:
\begin{gather}
u(x, t) = \sum_{j = 1}^{N} c_{j}(t)\phi_{j}(x)
\end{gather}
With \(\phi_{j}\) being defined as \ref{def-phi} and \(c_{j}: \mathbb{R} \rightarrow \mathbb{R} \) is equivalent to linear interpolation with respect to \(x\):
\begin{gather}
\hat{u}(x,t) =  u_{j} + \frac{(u_{j+1} - u_{j})(x-x_{j})}{x_{j+1} - x_{j}} 
\end{gather}
for \(x_{j} < x < x_{j+1}\) \cite{Bayen2015}. 
This can be shown by evaluating \(u(x, t)\) between two neigbouring \(\phi\) and \(x_{j} < x < x_{j+1}\):
\begin{gather}
u(x, t) = \phi_{j}(x) c_{j}(t) + \phi_{j+1}(x) c_{j+1}(t) \\
= \frac{x_{j+1} - x}{\Delta x} c_{j}(t) + \frac{x - x_{j}}{\Delta x} c_{j+1}(t) \\
= \frac{x_{j+1} - x}{\Delta x} u_{j} + \frac{x - x_{j}}{\Delta x} u_{j+1} \\
= \frac{(x_{j+1}u_{j} - xu_{j}) + (x u_{j+1}  - x_{j} u_{j+1})}{\Delta x} \\
= \frac{(u_{j+1} - u_{j})x + x_{j+1}u_{j} - x_{j} u_{j+1}}{\Delta x} \\
= \frac{(u_{j+1} - u_{j})x + x_{j}u_{j} + \Delta x u_{j} - x_{j} u_{j+1}}{\Delta x} \\
= u_{j} + \frac{(u_{j+1} - u_{j})x - x_{j}(u_{j+1} - u_{j})}{\Delta x} \\
= u_{j} + \frac{(u_{j+1} - u_{j})(x - x_{j})}{x_{j+1} - x_{j}}
\end{gather}

\pagebreak
\section{Source Code}
The source code described in chapter \ref{chap-impl} is available on github:

\href{https://github.com/FlorianGelb/FEM/tree/abgabe}{https://github.com/FlorianGelb/FEM/tree/abgabe}.
It can be downloaded using the following commands:
\begin{itemize}
 \item git clone [URL]
 \item git checkout abgabe
\end{itemize}
Here for example test.m can be executed to generate some figures similar to figure \ref{FIG-POD}.
The exact commit is 306d9ee27fbf2c8288b6ecd11cbd7bb22a7cc9ce.
To make sure this exact version is used, use the git reset command.



\newpage
\end{document}